\documentclass[12pt, b5paper,twoside]{tesi_upf_new}
%\documentclass[12pt, b5paper,twoside]{book}


%CODIFICATION
\usepackage[latin1]{inputenc}


%LENGUAGE
\usepackage[english,catalan]{babel}

%ONLY TO OBTAIN MARK BANK INDEX INDICATION B5
\usepackage[b5paper]{geometry}
\usepackage[cam,a4,center,frame]{crop}

%INCLUDE GRAPHICS AND THE LOGO OF THE UPF
\usepackage{graphicx}

%FONTS TIMES OR GARAMOND, 
\usepackage{times}
%\usepackage{garamond}

%COHERENCE ESTIMATION
\usepackage{graphicx}% Include figure files
\usepackage{dcolumn}% Align table columns on decimal point
\usepackage{bm}% bold math
\usepackage{latexsym}% 
\usepackage{lineno}   
\usepackage{array}
\usepackage{adjustbox}
\usepackage{amsmath,amsfonts,amssymb,amsthm}
\usepackage{newpxtext} 
\usepackage{cite}
\usepackage{booktabs}

% COMMENT THESE FOR TESI_UPF
\usepackage{hyperref}
\hypersetup{pdftex,colorlinks=true,allcolors=blue}
\usepackage{hypcap}

%SELD
\DeclareMathOperator{\LL}{\langle}
\DeclareMathOperator{\RR}{\rangle}
\usepackage{graphicx,url,times,booktabs, tabularx}
\usepackage{array}
\usepackage{booktabs}
\usepackage{algorithm, algorithmic}
\usepackage{color}
\usepackage{subcaption}

\usepackage[none]{hyphenat}
\tolerance=1
\emergencystretch=\maxdimen
\hyphenpenalty=10000
\hbadness=10000



\usepackage[titletoc]{appendix}


%WITHOUT HEADINGS: NO MODIFICATION
\pagestyle{plain}

%FOR THE INDEX OF SUBJECTS
\usepackage{makeidx}
\makeindex

%BIBLIOGRAPHY STYLE
\bibliographystyle{apalike}


%SELECT LANGUAGE
\selectlanguage{english}

%THE TABLE OF CONTENTS IS TITLE CONTENTS
%\addto\captionscatalan
  {\renewcommand{\contentsname}{\Large \sffamily Contents}}


%ADD YOUR DATA
%\title{Ambisonic domain methods for immersive audio production enhancement}
\title{Parametric analysis of ambisonic audio}
\subtitle{Contributions to methods, applications and data generation}
\author{Author: Andr�s P�rez L�pez}

% TODO UNCOMMENT FOLOWING
\thyear{2020}
\department{of Information and Communication Technologies}
\supervisor{Dr. Adan Garriga Torres\\Dr. Emilia G�mez Guti�rrez}

% TODO
\newcommand\todo[1]{\textcolor{red}{#1}}

% INTEGRAL D
\newcommand{\dif}{\mathop{}\!\mathrm{d}}




\begin{document}


\frontmatter

\maketitle

\cleardoublepage


%%%%%% Dedication

\noindent To my cat

\cleardoublepage

%%%%%% End dedication


%%%%%% Thanks
\noindent {\Large \sffamily Thanks}\\

First of all, I must thank my supervisors, Emilia and Adan, who have made possible the realization of this journey.

Thanks to all my friends from the Audio Team at Eurecat: Umut, Julien, Tim, Niklas, Toni, Gerard, Andr�s, John, Stefan, Marc. And also to  the people from Image and the Visualization teams. I spent last four years working with the finest of the companies.

Thanks also to all the great people I met at the MTG, including Pritish, Xavier, and specially Edu for the logistic and intellectual help. I would like to give public acknowledgment to Xavier Serra for his successful task of leading and gathering such unique international group of talented people.

Special thanks to Nikos and all the Signal Processing Lab crew for warmly welcoming me in Heraklion, and helping me in many different ways in your lovely place. I hope to come back there soon. 

Many thanks to Archontis and Tuomas for the visit opportunity, and for all the  facilities they gave me in my stay in this amazing place called Tampere. I take the opportunity to say hello and thanks to Kostas, Natalya, Annamaria, Toni, and all the nice people there.

Thanks to Max, who has been always very receptive and supportive on numerous occasions, and has indirectly contributed to the success of this project. 
Thanks also to Rafa, who decided to get on board for some unknown reason.

Thanks to all the friends I made and preserved in Barcelona in the last four years. Special thanks to my musician family from Balkumbia, Fanfarrai and Telewawachi Kilili; I learned a lot from you. Special acknowledgements for the good times with the people in Can Collet (best farm-school in town) and the Valencian oldies. 

Thanks a lot to Chedar for his company in the final steps. And, of course, million thanks to my family and to Marta for their unconditional love and support, powering all the way up here. \\


To all of you: thanks.

\cleardoublepage

%%%%%% End of thanks

%ABSTRACT IN TWO LEGUAGES. 150 words max!!!
\selectlanguage{english}
\section*{\Large \sffamily Abstract}


Due to the recent advances in virtual and augmented reality, ambisonics has emerged as the \textit{de facto} standard for immersive audio.
Ambisonic audio can be captured using spherical microphone arrays, which are becoming increasingly popular.
Yet, many methods for acoustic and microphone array signal processing are not specifically tailored for spherical geometries.
Therefore, there is still room for improvement in the field of automatic analysis and description of ambisonic recordings. 
In the present thesis, we tackle this problem using methods based on the parametric analysis of the sound field. 
Specifically, we present novel contributions in the scope of blind reverberation time estimation, diffuseness estimation, and sound event localization and detection. 
Furthermore, several software tools developed for ambisonic dataset generation and management are also presented. 


\selectlanguage{catalan}
\vspace*{\fill}
\section*{\Large \sffamily  Resum}



\textit{Ambisonics} ha esdevingut l'est�ndard d'�udio immersiu als �ltims anys, afavorit pels avan�aments en realitat virtual i augmentada.
L'�udio \textit{ambisonic} es pot obtindre mitjan�ant \textit{arrays} de micr�fons esf�rics, que s�n cada vegada m�s populars.
Tot i aix�, la majoria de m�todes ac�stics i de processament de senyal basats en \textit{arrays} de micr�fons no estan adaptats al cas espec�fic de geometries esf�riques.
Per tant, encara hi ha moltes possibilitats de millora en l'�mbit d'an�lisi autom�tica i descripci� de gravacions \textit{ambisonic}.
En la present tesi plantegem aquest problema basant-nos en l'an�lisi param�trica del camp ac�stic.
M�s concretament, presentem contribucions originals en les �rees d'estimaci� de reverberaci�, estimaci� de difusi� ac�stica, i detecci� i localitzaci� d'esdeveniments sonors.
Aix� mateix, presentem diverses eines de programari desenvolupades per la generaci� i manteniment de bases de dades d'�udio \textit{ambisonic}.

 
\vspace*{\fill}

\cleardoublepage
\selectlanguage{english}
%END OF ABSTRACT

%PREFACE. 
\section*{\Large \sffamily  Preface}

The present thesis has been carried out within the context of the Industrial Doctorate program from the Catalan government, as a collaboration between the Music Technology Group of the Pompeu Fabra University, and the Multimedia Unit of Eurecat, the Catalan Technology Center. 
It has been jointly supervised by Dr. Emilia G�mez and Dr. Adan Garriga. 

The industrial nature of the doctorate program implies a tendency towards the innovative application of the findings and conclusions generated during the development of the thesis. 
Therefore, one of the targets of the research performed in the context of this thesis is the stress on real-world problems, scenarios and technologies, with the aim of developing methods that could be easily adapted into applications with a high level of technology readiness. 
Moreover, there has been also an stress on research reproducibility. This has not been motivated only by an academic perspective; but also by the open research and open innovation paradigms present in our today's society.\\


The work presented here would not have been possible without the contributions from an excellent network of researchers. Apart from internal collaborations (UPF and Eurecat), two of the main contributions in the thesis followed two research visits to internationally recognized institutions: the Audio Research Group from Tampere University, Finland (Chapter~\ref{chap:rt60}), and the Signal Processing Group at the FORTH - Foundation for Research and Technology Hellas, Heraklion, Greece, (Chapter~\ref{chap:coherence}).

The alphabetical list of academic collaborators is: Eduardo Fonseca, Emilia G\'omez, Rafael Ib\'a�ez-Usach,  Julien de Muynke, Archontis Politis, Xavier Serra, and Nikolaos Stefanakis.




\cleardoublepage
%END OF PREFACE


%TABLE OF CONTENTS: REQUIRED
\tableofcontents

%lIST OF FIGURES; ONLY IF THERE ARE FIGURES
\listoffigures
%TO APPER THE LIST OF FIGURES IN THE TABLE OF CONTENTS 
\addcontentsline{toc}{chapter}{List of figures}

%LIST OF TABLES; ONLY IF THERE ARE TABLES
\listoftables
%TO APPEAR THE LIST OF TABLES IN THE TABLE OF CONTENTS
\addcontentsline{toc}{chapter}{List of tables}

%START THE TEXT
\mainmatter

%INSERT CHAPTERS
\chapter{Introduction}

%%%%%%%%%%%%%%%%%%%%%%%%%%%%%%%%%%%%%%%%%%%%%%%%
\section{Motivation}

Ambisonics is a spatial audio theory based on the directional decomposition of the sound field. Conceived in its primal form during the 70s \cite{gerzon1973periphony}, it was not until the 21st century, with a modern mathematical formulation \cite{daniel2000representation} and much more computational power available, that it definitely drew the attention of the research community. 


%\begin{figure}[t]
%  \includegraphics[width=\textwidth]{Figures/Introduction/hardware.jpg}
%    \caption{An ambisonic hardware processor, manufactured by Audio \& Design Recording Ltd. Source: \href{https://commons.wikimedia.org/w/index.php?curid=11813129}{Relen, Wikimedia Commons}}
%  \label{fig:hardware}
%\end{figure}

Nevertheless, the greatest contributor to the current interest in ambisonics has been the rise of Virtual Reality (VR) in recent years.
Although VR focuses primarily on visual cues, the immersive experience might be greatly enhanced by spatial audio \cite{begault20003}	. In this context, Ambisonics has been rapidly adopted as \textit{de facto} standard for spatial audio transmission, supposedly due to a variety of factors:

\begin{description}

  \item [Layout independence] As opposed to other audio spatialization techniques that rely on specific playback layouts, ambisonics makes use of an intermediate sound field representation, known as \textit{B-Format} (or just ambisonic audio). This representation, often referred to as \textit{scene-based}, can be then further processed to match any playback configuration. 

  \item [Recording device independence] Regardless of the specific characteristics of an ambisonic microphone, the recorded signal is usually converted into B-Format, which effectively represents the standard exchange format. 

  \item [Ease of manipulations] Signal-independent transformations of the ambisonic stream, and specifically rotations, are computationally inexpensive.

  \item [Binaural transformation] Spatial audio in VR is mostly consumed as binaural; methods for ambisonic to binaural conversion have been known for a long time \cite{noisternig20033d}. Furthermore, VR headsets can easily provide head rotation information, which can be used in combination with scene rotations to provide head-locked audio, which greatly improves localization accuracy and inmersiveness \cite{begault20003}. This is a key feature of ambisonics when compared to static binaural recordings.

\end{description}

Coming back to the Virtual Reality popularity, the following three events might be representative of the growth that it experienced in the second half of the 2010s: 
(i) the billionaire acquisition by Facebook of the VR headset manufacturer Oculus, in March 2014 \cite{facebookoculus}; 
(ii) the Time cover page devoted to VR: \textit{"The surprising joy of Virtual Reality. And why it's about to change the world"} (August 2015) \cite{time};
and (ii) M. Zuckerberg's invited talk at the \textit{Samsung Unpacked} event within the World Mobile Congress 2016 in Barcelona: \textit{"VR is the next platform where anyone can experience anything they want"} \cite{bbcnews}.


\begin{figure}[t!]
  \includegraphics[width=\textwidth]{Figures/Introduction/num_mics_ticks.png}
  \caption{Number of ambisonic microphones released in last years (from \cite{List_of_Ambisonic_hardware}). From left to right, the vertical lines correspond to (i) Oculus acquisition, (ii) Time cover page on VR, (iii) M. Zuckerberg's speech in MWC, and (iv) Jaunt announcement of shift towards AR.}
  \label{fig:nummics}
\end{figure}


\begin{table}[t!]
\centering
\caption{List of ambisonic microphones released in recent years (from \cite{List_of_Ambisonic_hardware}).}
\begin{tabular}{cccc}
  \toprule
Manufacturer & Model   & Year & Order \\
\midrule
MH Acoustics & EigenMike   & 2013                     & 4                         \\
Brahma       & (Brahma)                                                   & 2014                     & 1                         \\
Sennheiser   & Ambeo                                                      & 2016                     & 1                         \\
Twirling     & 720 VR    & 2016                     & 1                         \\
Zoom         & H2n                                                        & 2016                     & 1                         \\
Zylia        & ZM-1                                                       & 2017                     & 3                         \\
Twirling     & 720 Lite    & 2017                     & 1                         \\
Ricoh        &  TA-1        & 2017                     & 1                         \\
Nevaton      & Nevaton VR                                                 & 2017                     & 1                         \\
Rode         &  Rode NT-SF1 & 2018                     & 1                         \\
CoreSound    &  OctoMic     & 2018                     & 2                         \\
Zoom         & H3-VR                                                      & 2018                     & 1                         \\
Brahma       & Brahma 8                                                   & 2019                     & 2                         \\
Voyage Audio & Spatial Mic  & 2019    & 2  \\
\bottomrule
\end{tabular}
\label{tab:nummics}
\end{table}


Given this context, many microphone manufacturers and audio-related companies have followed the industry trend in the search for new markets. Only in the interval 2016-2019, 12 different ambisonic microphones have reached the market (Figure~\ref{fig:nummics}) --- a greater amount than all previous existing ambisonic microphones together. A comprehensive list of recent ambisonic microphone releases is shown in \ref{tab:nummics}.


At the present time, however, the high expectations in VR have significantly lowered, as shown in Figure~\ref{fig:vrfunding}. This is due to a variety of reasons, including lack of interesting content and the high production cost of the headsets \cite{fortune}. The change of focus of Jaunt (formerly one of the biggest VR film production companies) towards Augmented Reality (AR), as of October 2018, might be a paradigmatic example of this tendency \cite{theverge}. 

\begin{figure}[t!]
  \includegraphics[width=\textwidth]{Figures/Introduction/vr_funding.png}
  \caption{Venture investments in VR in the period 2014-2018. Adapted from \cite{fortune}.}
  \label{fig:vrfunding}
\end{figure}



In any case, the current high availability and affordability of ambisonic microphones brings new challenges from the signal processing perspective. More specifically, ambisonic microphones conform a subset of near-coincident spherical microphone arrays, a category which possesses some specific characteristics.

Although the VR momentum has also reached spherical microphone array processing, many challenges remain still open, and the number of research works specifically focusing on such geometry are low yet. 
Besides that, the growing interest in AR posses new problems related to acoustical signal processing. But since ambisonics is still the standard choice for immersive audio, existing solutions might be successfully adapted. 

Lastly, the advance in signal processing methods for ambisonics can give rise to applications that enhance the work of immersive audio producers, providing meaningful information about the recorded scenes and automating some of the repetitive tasks, thus allowing a more flexible and creative workflow.










%%%%%%%%%%%%%%%%%%%%%%%%%%%%%%%%%%%%%%%%%%%%%%%%
\section{Problem Description}
\label{sec:problemdescription}

The scientific context of the work developed in the present dissertation is depicted in Figure~\ref{fig:scheme1}, which has been inspired by \cite{jarrett2017theory}. 
As we can observe, there are two main topics related with B-Format audio: \textit{generation} and \textit{analysis}, with the signal flow going from the former to the latter.
Although the conceptual approach of the scheme might be very similar for any type of audio, the specific spatial information conveyed by the ambisonic signal places an emphasis on the informed analysis and description of the sound scene. \\

\begin{figure}[th]
  \includegraphics[width=\textwidth]{Figures/Introduction/SCHEME1.png}
  \caption{General scheme of the B-Format audio generation and analysis framework.  Solid lines represent audio signals, while outlined arrows refer to non-audio information.}
  \label{fig:scheme1}
\end{figure}


The problem of B-Format generation is mostly related with dataset generation, which is a common issue for many audio signal processing problems. 
In our case, we consider two different approaches to the data generation problem:
\begin{description}
	\item[Recording] Using spherical microphone arrays for the recording of ambisonic material.
	\item[Simulation] Using numerical methods for the simulation of acoustic scenes.
\end{description}
While recordings are by definition more similar to real scenarios, they are expensive to perform, and can only provide a limited set of parameter possibilities. Simulations, on the other hand, have the potential to cover any desired condition. 
Therefore, it is of our interest to consider the strengths of both signal generation paths.\\


B-Format analysis has been divided into three different categories:
\begin{description}
	\item[Acoustic Parameter Estimation] Low-level analysis of the sound field, which yields quantitative information about different acoustic parameters used to model the acoustic scene.
	\item[Signal Enhancement] Modification of the sound field in order to obtain a set of output signals with desired attributes. 
	\item[Scene Description] Textual representation of different high-level characteristics of the sound field.
\end{description}

The knowledge about the acoustic parameters of a sound field can be considered either as a goal by itself, or alternatively as a preprocessing step which complements the other analysis categories. Examples of typical estimated acoustic parameters are the \textit{Direction-of-Arrival} (DOA), the sound field diffuseness, or the reverberation time of the enclosure \cite{jarrett2017theory}.  

There are a variety of well known signal enhancement problems, including dereverberation \cite{braun2018speech}, source separation \cite{gannot2017consolidated}, or foreground-background segmentation \cite{stefanakis_foreground_2015}. 
As it has been shown, many of them benefit (or even depend) from the knowledge derived by the acoustic parameter estimation step.

Finally, under the scene description typology we can find a set of applications that provide abstract representations of the sound scene under analysis. Most of the recent research performed in this scope is grouped around the Detection and Classification of Acoustic Scenes and Events (DCASE) community\footnote{\url{http://dcase.community/community_info}}. 
Examples of the problems under consideration are acoustic scene classification \cite{Mesaros2018_DCASE}, sound event localization and detection \cite{Adavanne2018_JSTSP} or audio captioning \cite{drossos2017automated}.





%%%%%%%%%%%%%%%%%%%%%%%%%%%%%%%%%%%%%%%%%%%%%%%%
\section{Scientific Objectives}
\label{sec:objectives}


The list that follows concentrates the main scientific objectives to be developed on this thesis:

\begin{enumerate}
	\item To develop methods for the characterisation of acoustic parameters from recordings originated from ambisonic microphones. 
	\item To propose methodologies for sound event localization and detection in ambisonic domain which are grounded on spatial parametric analysis. 
	\item To contribute in the generation and storage of ambisonic sound scenes, for their usage in controlled experimental environments. 
\end{enumerate}



%%%%%%%%%%%%%%%%%%%%%%%%%%%%%%%%%%%%%%%%%%%%%%%%
\section{Outline}

\begin{figure}[t]
  \includegraphics[width=\textwidth]{Figures/Introduction/SCHEME1_NUMBERS.png}
  \caption{General scheme of the B-Format audio generation and analysis framework, including the thesis contributions in form of Chapter numbers.}
  \label{fig:scheme1_numbers}
\end{figure}

The present dissertation is organised as follows. 

\textbf{Chapter~\ref{chap:scientific}} introduces the basic concepts that will be developed throughout this thesis, including spherical harmonics and ambisonics, coherence estimation, parametric analysis or room acoustics. The Chapter also defines the signal models and the mathematical terminology.


Chapters~\ref{chap:rt60}, \ref{chap:coherence} and \ref{chap:seld2019} develop the most significant academic contributions of this thesis. 
\textbf{Chapter~\ref{chap:rt60}} presents a novel method for blind reverberation time in ambisonic recordings. To the best of our knowledge, this is the first method proposal specifically focusing on that problem.   The method is based on a Multichannel Auto-Regressive model of the late reverberation, which allows for an effective dereverberation of the ambisonic sound scene, and enables computation of the reverberation time from an estimation of the room impulse response. The evaluation metrics show a method performance similar to other state-of-the-art methods. 
\textbf{Chapter~\ref{chap:coherence}} analyses the response of tetrahedral microphone arrays, which are the simplest and most common form of ambisonic microphones, under spherically isotropic sound field. The analysis is performed using both simulated and recorded diffuse field, and the results quantify the differences between ideal and real values under a variety of conditions and estimators.  
In \textbf{Chapter~\ref{chap:seld2019}}, a complete system for Sound Event Localization and Detection of ambisonic sound scenes is described. The algorithm comprises two different parts. First, a parametric analysis is performed on the ambisonic signal. The analysis yields spatial localization and temporal activities of the sound events present in the scene. Then, each of those events is assigned to a class label by means of a deep-learning classifier. The method is able to perform in a similar way to the baseline system, while greatly improving its localization capabilities.

Finally, \textbf{Chapter~\ref{chap:data}} presents some libraries and software utilities developed throughout this thesis. All the code has been publicly released under open source licenses. The libraries include utilities for the creation of datasets, the storage and exchange of impulse response files in a standard way, and the implementation of convenience tools for acoustic and microphone array signal processing analysis. Although the libraries do not directly involve any scientific contribution, they can be a great help for scientific and innovative purposes; given the industrial nature of this thesis, we have considered relevant to include them in the present dissertation.\\

In order to provide a schematic representation of the thesis structure and scope, Figure~\ref{fig:scheme1_numbers} features the problem structure stated in Figure~\ref{fig:scheme1}, with the addition of the Chapter numbers with the contributions of this thesis. 



%%%%%%%%%%%%%%%%%%%%%%%%%%%%%%%%%%%%%%%%%%%%%%%%
\section{Contributions}


In the following list we show the main scientific contributions of this dissertation, organised by Chapters:

\begin{itemize}

	\item Chapter 3:\\
	\textbf{"Blind reverberation time estimation from ambisonic recordings"}.
	\underline{A. P\'erez-L\'opez}, A. Politis and E. G\'omez.
	Submitted to \textit{IEEE 22nd International Workshop on Multimedia Signal Processing, 2020.}
	
	\item Chapter 4:\\
	\textbf{"Analysis of spherical isotropic noise fields with an A-Format tetrahedral microphone"}.
	\underline{A. P\'erez-L\'opez} and N. Stefanakis.
	\textit{The Journal of the Acoustical Society of America 146.4 (2019): EL329-EL334.}

	\item Chapter 5:\\
	
	\textbf{"PAPAFIL: a low complexity sound event localization and detection method with parametric particle filtering and gradient boosting".}
	\underline{A. P\'erez-L\'opez} and R. Iba�ez-Usach.
	Submitted to \textit{Detection and Classification of Acoustic Scenes and Events 2020 Workshop (DCASE2020)}.
	
	\textbf{"A hybrid parametric-deep learning approach for sound event localization and detection".}
	\underline{A. P\'erez-L\'opez}, E. Fonseca and X. Serra.
	In \textit{Proceedings of the Detection and Classification of Acoustic Scenes and Events 2019 Workshop (DCASE2019)}.
	
	
	\item Chapter 6:\\
	\textbf{"Ambiscaper: A Tool for Automatic Generation and Annotation of Reverberant Ambisonics Sound Scenes".}\\
	\underline{A. P\'erez-L\'opez}.
	In \textit{16th International Workshop on Acoustic Signal Enhancement (IWAENC). IEEE, 2018.}
	
	\textbf{"Ambisonics directional room impulse response as a new convention of the spatially oriented format for acoustics".}
	\underline{A. P\'erez-L\'opez} and J. De Muynke.
	In \textit{Audio Engineering Society Convention 144. Audio Engineering Society, 2018.}
	
	\textbf{"pysofaconventions, a Python API for SOFA".}
	\underline{A. P\'erez-L\'opez}.
	In \textit{Audio Engineering Society Convention 148. Audio Engineering Society, 2020.}
	
	\textbf{"A Python library for Multichannel Acoustic Signal Processing".}
	\underline{A. P\'erez-L\'opez} and A. Politis.
	In \textit{Audio Engineering Society Convention 148. Audio Engineering Society, 2020.}
	
\end{itemize}

Moreover, although not strictly aligned with the research direction of this thesis, the author has collaborated in the following publications:

\begin{itemize}
	\item \textbf{"Sound event localization and detection based on CRNN using dense rectangular filters and channel rotation data augmentation".}
	F. Ronchini, D. Arteaga and \underline{A. P\'erez-L\'opez}.
	Submitted to \textit{Detection and Classification of Acoustic Scenes and Events 2020 Workshop (DCASE2020)}.
	
\end{itemize}


Finally, as a result of the development of this thesis, the following open-source libraries have been implemented and released. All of them are available through the author's GitHub page \cite{andresgithub}:
\begin{itemize}

	\item \href{https://github.com/andresperezlopez/ambisonic_rt_estimation}{ambisonic\_rt\_estimation}
	
	\item \href{https://github.com/andresperezlopez/DCASE2020}{DCASE2020}
	
	\item \href{https://github.com/andresperezlopez/DCASE2019_task3}{DCASE2019\_task3}
	
	\item \href{https://github.com/andresperezlopez/masp}{masp: Multichannel Acoustic Signal Processing library}

	\item \href{https://github.com/andresperezlopez/pysofaconventions}{pysofaconventions}
	
	\item \href{https://github.com/andresperezlopez/AmbisonicsDRIR}{AmbisonicsDRIR}	

	\item \href{https://github.com/andresperezlopez/ambiscaper}{AmbiScaper} 
	


\end{itemize}




\chapter{Background}

\section{Conventions}
\subsection{Reference system}

In what follows, we will make use of a right-handed coordinate system, where the positive \textit{x}-axis points towards the \textit{front}, the positive \textit{y}-axis points towards the \textit{left}, and the positive \textit{z}-axis points towards the \textit{zenith} (North Pole).

Any position in the unit sphere may be described in spherical coordinates by two angles: the \textit{inclination} angle $\vartheta$, which accounts for the aperture with respect to the \textit{z}-axis, and the \textit{azimuth} angle $\varphi$, which represents the counter-clockwise angle with respect to the \textit{x}-axis from the top-view. 
The value ranges are $0 \leq \vartheta \leq \pi$ for the inclination, and $0 \leq \varphi \leq 2\pi$ for the azimuth. 

Table~\ref{tab:cartesian} shows the spherical coordinate values for some reference points on the unit sphere.
Notice that the poles ($\vartheta = �\pi$) are a special case for the spherical coordinate system -- in that case, the azimuth angle is not defined. \\


\begin{table}[!htbp]
\centering
\caption{Cartesian and spherical representation of characteristic points along the unit sphere.}
  \begin{tabular}{cccc}
  \toprule
    Position & Cartesian & $\vartheta$ & $\varphi$ \\
\midrule
	front & $[1, 0, 0]$ & $\pi/2$ & $0$ \\
	back & $[-1, 0, 0]$ & $\pi/2$ & $\pi$ \\
	left & $[0, 1, 0]$ & $\pi/2$ & $\pi/2$ \\
	right & $[0, -1, 0]$ & $\pi/2$ & $-\pi/2$ \\
    \textit{zenith} & $[0, 0, 1]$ & $0$ & * \\
    \textit{nadir}  & $[0, 0, -1]$ & $\pi$ & * \\
    \bottomrule
  \end{tabular}
\label{tab:cartesian}
\end{table}


The transformation between spherical and cartesian coordinate systems is given by the following relationship:
\begin{equation}
	\begin{aligned}
		x = & \cos{\varphi} \sin{\vartheta}\\
		y = & \sin{\varphi} \sin{\vartheta}\\
		z = & \cos{\vartheta}\\
	\end{aligned}
\end{equation}

The \textit{elevation} angle $\theta$ provides an alternative way of describing the relationship with respect to the \textit{z}-axis. $\theta$ is defined as  the aperture with respect to the \textit{xy}-plane, with positive values towards the positive \textit{z}-axis. The relationship between elevation and inclination angles is:
\begin{equation}
	\theta = \pi/2 - \vartheta
\end{equation}


For the sake of compactness, a point in the unit sphere will be often represented by $\Omega = (\vartheta, \varphi)$.\\

Given the periodic nature of the azimuth angle, the descriptive statistic operations applied to $\vartheta$ will refer to the $2\pi$-\textit{periodic} version or the operator; this situation does not affect the inclination/elevation coordinate.


\subsection{Nomenclature}

Through the document, we refer to time-domain signals with lowercase, e.g. $x(t)$, with $t$ as the time index. 

Time-domain signals transformed by the Short-Time Fourier Transform (STFT) are represented with uppercase, e.g. $X(k,n)$, where $k$ is the frequency bin index, and $n$ the time frame index. 

Multichannel signals are in general denoted by a subscript variable index, usually with the letter $m$; for example, $x_m(t)$ or $X_m(k,n)$.
Signals with an integer subscript index, such as $x_0(t)$, represent a specific channel of the corresponding multichannel signal.

In the context of ambisonic, subscripts and superscripts are used in signal names with a specific meaning; check Section~\ref{sec:ambisonics} for a detailed explanation.

Vector notation is represented with boldface characters, e.g. $\bm{X}(k,n)$.  When used, the way to construct the vectors will be specified. 

\todo{revise all vector notations}




\section{Spherical Harmonics}

\subsection{Definition}

Spherical harmonics are continuous functions defined on the sphere surface. Due to their mathematical properties, any spherical function can be decomposed as a combination of spherical harmonics, in what is known as the \textit{Spherical Harmonics Expansion} [Jarrett book, page 17].




%The spherical harmonic of \textit{order} $\ell>0$ and \textit{degree} $m$, with $|m| <= \ell$  in the direction \Omega is defined as:
Many different spherical harmonic definitions exist in the literature, with minor variations among them. In the following, we will use the real-valued, fully normalized spherical harmonics as defined by [Zotter]: 

\begin{equation}
%ambisonics book 186
	Y_n^m(\varphi, \vartheta) = N_n^{|m|} P_n^{|m|}\cos(\vartheta) \Phi_m(\varphi), 
	\label{eq:sphericalharmonics}
\end{equation}

where the \textit{normalization factor} $N_n^m$ is:

\begin{equation}
%ambisonics book 186
	N_n^m = (-1)^m \sqrt{\frac{2n+1}{2} \frac{(n-m)!}{(n+m)!}}
\end{equation}

the \textit{Legendre polynomials} $P_n^{m}$ are defined as: 

\begin{equation}
%ambisonics book 185
	P_{n+1}^{m} =
	\begin{cases}
 		\frac{2n+1}{n-m+1} x P_n^{m},  &\text{for } n = m,  \\
 		\frac{2n+1}{n-m+1} x P_n^{m} - \frac{n+m}{n-m+1}P_{n-1}^{m} &\text{else}, \\
 	\end{cases}
\end{equation}

with $P_{n}^{n} = \frac{(-1)^n(2n)!}{2^nn!}\sqrt{1-x^{2^n}}$
and the initial term $P_{0}^{0} = 1$, 

and $\Phi_m$ is the azimuthal part of the spherical harmonics: 

\begin{equation}
%ambisonics book 176
	\Phi_m(\varphi) = \frac{1}{\sqrt{2\pi}}
	\begin{cases}
    	\sqrt{2} \sin(|m|\varphi),  &\text{for } m < 0,  \\
    	1,  & \text{for } m = 0,  \\
    	\sqrt{2} \cos(m\varphi),  & \text{for } m > 0.  \\
  	\end{cases}
\end{equation}


One of the properties of the spherical harmonics is orthonormality on the sphere surface:

\begin{equation}
	\int_{\mathbb{S}^2} Y_n^m(\varphi, \vartheta) Y_{n'}^{m'}(\varphi, \vartheta) \dif cos{\vartheta} \dif \varphi = \delta_{nn'} \delta_{mm'},
	\label{eq:orthonormality}
\end{equation}

where $\delta_{xy}$ represents the Kronecker delta operator:
\begin{equation}
	\delta_{xy} = \begin{cases}
		1,  &\text{if } x = y,\\
		0,  &\text{else}.
	\end{cases}
\end{equation}

The spherical harmonics depend on the \textit{order} $n \geq 0$ and the \textit{degree} $m$,  $|m| \leq n$ for each value of $n$. In practice, the maximum order $N$, $n \leq N$ determines the spatial resolution of the sound field expansion.\\

Through the spherical harmonic expansion, any sound field may be represented with a limited spatial resolution by the finite combination of all spherical harmonics up to order $N$. 
For a given order $n$, the number of spherical harmonic functions is $2n+1$. With the accumulation of all orders up to $N$, the total number of spherical harmonics is given by $M = (N+1)^2$.
Figure~\ref{fig:sphericalharmonics} depicts all spherical harmonics from orders 0 to 3.

\begin{figure}[hbt]
  \includegraphics[width=\textwidth]{Figures/ScientificBackground/Spherical_Harmonics_deg3.png}
  \caption{Spherical harmonics up to order $N=3$. The rows correspond to the spherical harmonics of a given order $n$, and the columns span all possible degree values.}
  \label{fig:sphericalharmonics}
\end{figure}


\subsection{Spherical array processing}

Let us consider a sound field captured with a spherical microphone array, which contains $Q$ capsules distributed around a spherical surface of radius $R$ at the positions $\Omega_q, 1 \leq q \leq Q$. 
The captured frequency-domain signals $X_q(k)$ can be represented as the spherical harmonic domain signals $X_n^m(k)$ through the spherical harmonic transform of order $n$ and degree $m$ (Moreau et al., 2006):

\begin{equation}
	X_n^m(k) = \sum_{q=1}^{Q} X_q(k) Y_n^m(\Omega_q) \Gamma_n(kR),
	\label{eq:a2b}
\end{equation}
\todo{is that actually valid? or Y should be the complex-valued spherical harmonics?}

where the term $\Gamma_n(kR)$ models the radial transfer function, and depends on the microphone geometry[SOME REFS: moreau, rafaely]. There are several possible sampling schemes of capsules along the sphere, each one having different properties; the reader is redirected to [rafaely] for a deeper insight. \\

By using this model, the maximum spherical harmonic order $N$ that can be retrieved with negligible spatial aliasing depends on the number of microphone capsules [moreau]:
\begin{equation}
	N \geq (Q + 1)^2.
\end{equation}

Furthermore, the sphere radius $R$ has also an effect on the operational bandwidth of the microphone. More precisely, for a given spherical harmonic order $n$, the maximum aliasing-free operational frequency is given by [moreau, rafaely]:
\begin{equation}
	f_{max} = \frac{n c} {2 \pi R},
	\label{eq:falias}
\end{equation}

with $c$ being the sound speed.

\todo{moreau has c/2Rgamma}.




\section{Ambisonics}
\label{sec:ambisonics}

\subsection{Ambisonics Theory}

Ambisonics is a spatial sound recording and playback technology initially developed during the 1970's, and further expanded into its modern formulation around the 2000s [ZOTTER, page 53].  
Ambisonics is based on the idea of decomposing a sound field into its spherical harmonic representation. 

Originally, the decomposition was limited to first-order spherical harmonics, mainly due to practical limitations [CITE GERZON], as the so-called \textit{First Order Ambisonics} (FOA). The technique was later formalized for arbitrary spherical harmonic orders, known as \textit{Higher Order Ambisonics} or HOA [CITE DANIEL].
In general, with the term Ambisonics we will be referring to the latter definition.\\

\subsubsection{Ambisonic encoding}

Let us consider a sound field composed of a point sound source $S$ located in far-field at the angular position $\Omega_s$. The sound pressure at the coordinate origin $P$ can be expressed in terms of the spherical harmonic expansion of order $N$ as: \todo{check equation, find references, how to explain the domain? extend also to multiple sources by superposition}
\begin{equation}
	P = \sum_{n=0}^{N} \sum_{m=-n}^{n} Y_n^m(\Omega_s) S
	\label{eq:encoding}
\end{equation}


The ordered set of values of all spherical harmonics up to order $N$, evaluated at the source position, is known as the \textit{ambisonic coefficients}:
\begin{equation}
	Y_n^m(\Omega_s) = [Y_0^0(\Omega_s), Y_1^{-1}(\Omega_s),  \ldots ,  Y_N^N(\Omega_s)]
	\label{eq:sphericalharmonicvector}
\end{equation}

Furthermore, the process of multiplying the signal $S$ by the ambisonic coefficients is known in the literature as the \textit{ambisonic encoding}. The resulting signal vector is usually referred to as the \textit{ambisonic} (or \textit{B-Format}) signal $S_n^m$:
\begin{equation}
	S_n^m = Y_n^m(\Omega_s) S
\end{equation}


Although the term \textit{B-Format} was initially introduced as an alternative name for first-order ambisonic signals [gerzon, tesis de daniel], it is nowadays common to use it as a synonim of ambisonic signals, without any order restriction. We will use the latter acception in what follows.

Historically, the name \textit{B-Format} was used as an opposite of \textit{A-Format}, which describes the signals recorded by a tetrahedral microphone array [cite gerzon?]. The tetrahedron is the simplest and most common form of spherical microphone arrays (\textit{ambisonic microphones}) with uniform capsule distribution. Again, the term \textit{A-Format} is also currently employed for referring to the signals recorded by any spherical microphone array, regardless of the number or arrangement of capsules.

Likewise, the process of signal conversion from the spatial domain (microphone capsules) to the spherical harmonic domain (ambisonic signals), as in Eq.~\ref{eq:a2b}, is known as \textit{A-B conversion}. A number of different approaches have been developed for this process, and the interested reader is referred to [\todo{find ref}] for more information.

In practice, there are two alternative ways to generate ambisonic signals. The first one is the \textit{synthesis}, based on the direct application of ambisonics encoding (Eq.~\ref{eq:encoding}) to a monophonic signal. The second one is the \textit{recording} with a spherical microphone array, followed by the aforementioned domain conversion. \\


\subsubsection{Ambisonic Decoding}
Conversely, the sound field reconstruction is performed by the \textit{ambisonic decoding} operation. This process is equivalent to weight-and-sum beamforming in the spherical harmonic domain, and it is sometimes also referred to as the \textit{virtual microphone} technique [ambisonic book].

Let us consider a loudspeaker located at the angular position $\Omega_p$. In accordance with Eq.~\ref{eq:encoding}, the signal feed $P$ is \textit{decoded} from the ambisonic signal as:
\begin{equation}
	P = \sum_{n=0}^{N} \sum_{m=-n}^{n} Y_n^m(\Omega_s) S Y_n^m(\Omega_\ell)  \alpha_n 
	\label{eq:decoding}
\end{equation}

where $\alpha_n$ is a weighting factor which accounts for the beam directivity.
There are several standard weightings used for different purposes; their values are shown in Table~\ref{tab:alpha}, and the first-order directive patterns are plotted in Figure~\ref{fig:alpha}.

\begin{table}[t]
\caption{Ambisonic decoding: standard values of $alpha_n$ weightings. Extracted from \cite{daniel2000representation}.}
\begin{center}
\begin{tabular}{cccccc}
\toprule
Decoding & $N$ & \multicolumn{4}{c}{$n$} \\ 
&  & 0 & 1 & 2 & 3 \\
\midrule
\textit{basic} & 0 & 1  \\
	& 1 & 1 & 1  \\
 	& 2 & 1 & 1 & 1 \\
 	& 3 & 1 & 1 & 1 & 1 \\
\midrule
\textit{max-rE} & 0 & 0.577  \\
	& 1 & 0.775 & 0.4  \\
 	& 2 & 0.861 & 0.612 & 0.305 \\
 	& 3 & 0.906 & 0.732 & 0.501 & 0.246 \\
\midrule
\textit{in-phase} & 0 & 0.333  \\
	& 1 & 0.5 & 0.1  \\
 	& 2 & 0.6 & 0.2 & 0.029 \\
 	& 3 & 0.667 & 0.286 & 0.071 & 0.008 \\
\bottomrule
\end{tabular}
\label{tab:alpha}
\end{center}
\end{table}

\begin{figure}[htbp]
	\begin{center}
	\begin{minipage}[b]{0.9\linewidth}
		\centerline{\includegraphics[width=\textwidth]{Figures/Introduction/basic.png}}
	\end{minipage}
	\begin{minipage}[b]{0.9\linewidth}
		\centerline{\includegraphics[width=\textwidth]{Figures/Introduction/maxre.png}}
	\end{minipage}
		\begin{minipage}[c]{0.9\linewidth}
		\centerline{\includegraphics[width=\textwidth]{Figures/Introduction/inphase.png}}
	\end{minipage}
	\caption{Directive patterns of first-order ambisonic decoding.}
	\label{fig:alpha}
	\end{center}
\end{figure}


The decoding equation \ref{eq:decoding} can be written in matrix form as:
\begin{equation}
	P = S_n^m {Y_n^m (\Omega_p)}^T \alpha_n
\label{eq:decodingequation}
\end{equation}

where the superscript $T$ represents the matrix transposition. 
This equation can be extended to the usual case of decoding to a loudspeaker array, comprised of $L$ loudspeakers located at the positions $\Omega_L = [\Omega_{p_1}, \ldots, \Omega_{p_L}]$ . In such case, the loudspeaker feed vector $P_L$ can be written as:
\begin{equation}
	P_L = S_n^m D,
\label{eq:decodingequation2}
\end{equation}
where 
\begin{equation}
	D = \text{diag}(\alpha_n) [{Y_n^{m}(\Omega_{p_1})}^T, \ldots, {Y_n^{m}(\Omega_{p_L})}^T]
\end{equation}

is a $M \times L$ matrix known as the \textit{decoding matrix}, and $\text{diag}(\alpha_n)$ is a diagonal matrix of size $M$ containing the values of $\alpha_n$ along the main diagonal. 
Although the matrix $D$ is frequency-independent and depends solely on the loudspeaker array geometry, in practical scenarios it is usual to include frequency-dependent weightings, $\alpha_n(k)$, to improve the broadband sound field reconstruction \cite{daniel2000representation}.

Furthermore, sound field reconstruction with Eq.~\ref{eq:decodingequation2} is only possible when the loudspeakers are evenly located on the 3D space; in other words, the speaker layout must take the form of one of the five \textit{Platonic solids}: tetrahedron, cube, octahedron, dodecahedron or icosahedron.
Provided that this condition is usually difficult to fulfil in real scenarios, there are several methods which allow ambisonic decoding for such \textit{irregular} layouts. One of the most commonly used is the AllRAD method \todo{cite zotter}. AllRAD proposes a two step decoding: first, the ambisonic signal is decoded to a nearly-uniform layout of virtual speakers. Then, the signals of the virtual speakers are further distributed into the real speakers by the \textit{Vector-Based Amplitude Panning (VBAP)} method \todo{cite pulkki}.


% The reader is referred to [zotter? daniel? idhoa?] for more information about the vast field of study of ambisonic decoding.






\subsection{Practical considerations}

Due to historical and practical reasons, there are two aspects that must be taking into account when working with ambisonic signals: \textit{channel normalization} and \textit{channel ordering}. 
In the following, the term \textit{channels} will be used as a synonym for spherical harmonics, as they are usually referred to in sound engineering contexts\footnote{In fact, ambisonic signals are inherently multichannel, even though each channel corresponds to a spherical harmonic, and not to a loudspeaker feed as in traditional \textit{channel-based} audio.}.  \\


\subsubsection{Channel normalization}
Let us consider the spherical harmonics $Y_n^m(\Omega)$ as defined in Eq.~\ref{eq:sphericalharmonics}. Due to the orthonormal property showed in Eq.~\ref{eq:orthonormality}, they follow the \textit{fully 3d normalized} or \textit{N3D} channel normalization convention. \todo{what about the 1/sqrt(4pi)???}. 

Alternatively, the \textit{Schmidt 3d semi-normalized} or \textit{SN3D} [daniel] convention is also of widespread usage. The conversion between \textit{N3D} and \textit{SN3D} is driven by the following expression:
\begin{equation}
	{Y_n^m(\Omega)}^{\text{(N3D)}} = \sqrt{2n+1} {Y_n^m(\Omega)}^{\text{(SN3D)}}
\end{equation}

\textit{MaxN} is another existing convention. It defines all spherical harmonics as having a maximum absolute value of 1: 
\begin{equation}
	\max_{\Omega} |{Y_n^m(\Omega)}^{\text{(MaxN)}}| = 1, \forall (n, m)
\end{equation} 

Finally, the \textit{Furse-Malham} (or \textit{FuMa}) normalization only differs from \textit{Max-N} in the scaling of the zero-th order component: 
\begin{equation}
	{Y_n^m(\Omega)}^{\text{(FuMa)}} = \begin{cases}
		1 / \sqrt{2},  &\text{if } n = 0,\\
		{Y_n^m(\Omega)}^{\text{(MaxN)}},  &\text{else}.
	\end{cases}	
\end{equation} 

Each of the normalization schemes has its own particularities. For instance, \textit{N3D} is the most mathematically straightforward, and spherical harmonics defined in that way can be directly used for both encoding and decoding (as in Eqs~\ref{eq:encoding} and Eq.~\ref{eq:decoding}) -- however, from a sound engineer point of view, other normalization schemes with maximum values below the unity might be preferred, such as \textit{SN3D}. Besides this, \textit{FuMa} has been historically the default normalization \todo{[gerzon?]}, while the more modern \textit{N3D} and \textit{SN3D} were popularized after Daniel \todo{[daniel]}. \\

\begin{figure}[hbt!]
  \includegraphics[width=\textwidth]{Figures/ScientificBackground/normalization.png}
  \caption{Maximum value of each ambisonic channel up to order 5, for all different normalization schemes. Image from T. Carpentier [\todo{cite}].}
  \label{fig:normalization}
\end{figure}


As a summary, Figure~\ref{fig:normalization} displaysthe different normalization schemes. The reader is referred to \todo{[thibaut]} for an extensive review on the topic.



\subsubsection{Channel ordering}

Channel ordering refers to the manner in which spherical harmonics, inherently organized in the 2D space by dimensions $n$ and $m$, are sorted into a one-dimensional vector. 

The \textit{ACN} (from \textit{Ambisonic Channel Number}) scheme follows from the mathematical description given in Eq.~\ref{eq:sphericalharmonicvector}. The spherical harmonics are first ordered by ascending order $n$ and, inside each order, by ascending degree $m$. The index of a given channel $i \in [0 \ldots M-1]$ can be thus obtained by the following relationship:
\begin{equation}
	i = n^2 + n + m 
\end{equation}

Historically, first-order ambisonic audio has followed what it might be called \textit{traditional B-Format} channel ordering [ambisonics in multichannel broadcasting and video]. 
By following this scheme, the four channels of a FOA signal $S_n^m$ are referred to by the axis where the corresponding spherical harmonic steers, plus the name $W$ for the zeroth order component:
\begin{equation}
	{S_n^m(\Omega)}^{\text{(FuMa CO)}} = [W, X, Y, Z]
\label{eq:fumaordering}
\end{equation}
with:
\begin{equation}
\begin{aligned}
	&W = S_0^0(\Omega) \\
	&X = S_1^1(\Omega) \\
	&Y = S_1^{-1}(\Omega) \\
	&Z = S_1^0(\Omega)
\end{aligned}	
\end{equation}

This nomenclature was extended to second and third order by Malham, and is currently known as the \textit{Furse-Malham} or \textit{FuMa} channel ordering. The channel names use all english alphabet letters from K to Z in third order and, although there would be enough letters to go up to fourth order, the unconvenience of the system was evident [Higher order Ambisonic systems].
Figure~\ref{fig:normalization} shows the equivalence between \textit{FuMa} (\textit{"letter code"}) and \textit{ACN} channel names.\\


In practice, there exist two main combinations of channel normalization and ordering schemes:
\begin{itemize}
  \item The \textit{classical} approach, usually limited to first-order ambisonics, which uses \textit{FuMa} normalization and channel ordering\footnote{In general, it may be expected that \textit{early} ambisonic material follow these conventions without any explicit mention to them.}.
  \item The \textit{modern} approach, inspired by the \textit{ambix} file format [cite ambix], with \textit{SN3D} normalization and \textit{ACN} channel ordering.
\end{itemize}

Anyhow, the \textit{classical B-Format} channel naming and ordering is still widely used when referring to first-order ambisonics. 
%In what follows, we will use indistinctly both \textit{classical} and \textit{ACN} conventions. \todo{is that true?}


\section{Parametric Spatial Audio Analysis}

Trough parametric analysis, sound fields may be described in terms of a small amount of sound sources and associate parameters. Such representation might reduce to a great extent the complexity of processing methods [book jarrett].

One of the most successful sound field parametric models is DirAC[pulkki2007 dirac]. Originally conceived as a method for impulse response processing and spatial sound reproduction [merima2005], it has been widely used in many different audio-related problems [\todo{find some references!}].

DirAC (acronym for \textit{Directional Audio Coding}) is a perceptually motivated time-frequency (TF) domain method, based on the assumption that any sound field may be reproduced with high perceptual quality by considering two parameters: the sound field diffuseness and the most prominent sound \textit{Direction-of-arrival} (DOA) [cite parametric book]. \\



Let us consider a \textit{N3D}-normalized first-order ambisonic signal in time-frequency domain, $S_n^m(k, n)$. 
For the sake of clarity, we will use in this section \textit{FuMa} channel notation and ordering (Eq.~\ref{eq:fumaordering}):
\begin{equation}
	S_n^m(k, n) = [W(k, n), X(k, n), Y(k, n), Z(k, n)]
\end{equation}

Given this representation, we can express the \textit{pressure} $P(k,n)$ of the sound field as:
\begin{equation}
	P(k,n) = W(k,n)
\end{equation}
as well as the sound \textit{pressure-gradient} (or \textit{velocity}) $\pmb{U}(k,n)$ as: 
\begin{equation}
	\pmb{U}(k,n) = - \frac{1}{\rho_0 c} [X(k, n), Y(k, n), Z(k, n)], 
\end{equation}
where $\rho_0$ is the mean density of the medium, and $c$ is the speed of sound.\\ 

The \textit{active intensity} $\pmb{I}(k,n)$, defined as the amount of transmitted acoustic energy, can be expressed in terms of sound pressure and velocity [fahy2002]:
\begin{equation}
	\begin{aligned}
	\pmb{I}(k,n) &=  \Re\{P^*(k,n)\pmb{U}(k,n)\} \\
	&= - \frac{1}{\rho_0 c}\Re\{W^*(k,n)[X(k,n),Y(k,n),Z(k,n)]\},
	\end{aligned}
\end{equation}
where $^*$ represents the complex conjugate operator. \\

An estimate of the instantaneous DOA $\Omega(k,n)$ can be extracted from the intensity vector, interpreting each of its time-frequency bins as a point in the cartesian space. Effectively, the sound propagation direction is the opposite to the observed arrival direction. 
\begin{equation}
	\Omega(k,n) = \angle(-\pmb{I}(k,n)),
\label{eq:doa}
\end{equation}
with $\angle$ representing the spherical angle operator of a cartesian vector. The result of this computation must be understood as the direction of the net energy flow, which in the case of a single plane-wave will correspond to the source position. \\

Another useful parameter is the \textit{energy density} $E(k,n)$ [stazial 1996]:
\begin{equation}
	\begin{aligned}
		E(k,n) &= \frac{|P(k,n)|^2 + \|\pmb{U}(k,n)\|^2}{2\rho_0 c^2} \\ 
		&=  \frac{|W(k,n)|^2 + \| [X(k,n), Y(k,n), Z(k,n)] \|^2}{2\rho_0 c^2}.
	\end{aligned}
	\label{eq:energydensity}
\end{equation}\\

Finally, the \textit{diffuseness} $\Psi(k,n)$ can be computed from the sound intensity and  energy density [merimaa2005]:
\begin{equation}
	\begin{aligned}
		\Psi(k,n) &= 1 - \frac{ \| \langle \pmb{I}(k,n) \rangle \| }{ c \langle E(k,n) \rangle } \\
		&= 1 - 2\frac{ \| \langle \Re\{W^*(k,n)[X(k,n),Y(k,n),Z(k,n)]\} \rangle \| }{ \langle |W(k,n)|^2 + \| [X(k,n), Y(k,n), Z(k,n)] \|^2 \rangle },
	\end{aligned}
\label{eq:psidefinition}
\end{equation}
 
 where the symbols $\langle \cdot \rangle$ represent the expectation operator, which is usually implemented as time-domain averaging. \todo{check equation}

Even though Eq.~\ref{eq:psidefinition} (known as \textit{DirAC's diffuseness}) is one of the most common ambisonic diffuseness estimators, several alternative formulations exist. 
Other diffuseness estimation procedures include the \textit{coefficient of variation method} [ahonen, pulkki, del galdo] and the more recent \textit{COMEDIE} estimator [epain, craig]. 

In any case, in what follows, the term \textit{diffuseness} and the symbol $\Psi$ will refer by default to Eq.~\ref{eq:psidefinition}. 

As a mathematical convenience, we will define the \textit{B-Format coherence} as the complement of the diffuseness:
\begin{equation}
	\Delta(k,n) = 1 - \Psi(k,n) 
	\label{eq:delta}
\end{equation}

In conclusion, Figure~\ref{fig:dirac} plots the spectrograms of the DOA $\Omega(k,n)$ and diffuseness $\Psi(k,n)$ of a FOA recording, which consists of a sound source located at the front, plus a moderate amount of reverberation and background noise. 

\begin{figure}[h!]
	\begin{center}
	\includegraphics[width=\textwidth]{Figures/Introduction/spectrograms.png}
	\caption{Parametric time-frequency spatial audio analysis of a first order ambisonic recording. From top to bottom: 1.) Magnitude spectrogram of the omnidirectional channel. 2.) and 3.) Azimuth and elevation of the estimated instantaneous narrowband DOAs $\Omega(k,n)$. 4.) Instantaneous narrowband diffuseness $\Psi(k,n)$.}
	\label{fig:dirac}
	\end{center}
\end{figure}




\section{Spatial Coherence Analysis}
\todo{put this chapter in context or something}\\


In the context of microphone array signal processing, diffuseness is commonly estimated through the
\textit{Magnitude Squared Coherence} (MSC) \cite{elko_spatial_2001} between two frequency-domain signals $S_1$ and $S_2$, as a function of the
\textit{wavenumber} $k$ and the microphone distance $r$:
\begin{equation}
    \text{MSC}_{12}(k r) =
	\frac{|\left\langle S_1(k r) S_2(k r)^* \right\rangle|^2}
	{\left\langle|S_1(k r)|^2\right\rangle \left\langle|S_2(k r)|^2\right\rangle},
    \label{eq:MSC}
\end{equation}
where the $\left\langle \cdot \right\rangle$ operator represents the temporal
expected value, and $^*$ defines the complex conjugate operator. In the case of spherical isotropic noise fields, Eq.~(\ref{eq:MSC})
can be expressed in terms of microphone directivity patterns
$T(\phi,\theta,k r)$ as \cite{elko_spatial_2001}:

\begin{equation}
	\begin{aligned}
&\text{MSC}_{12}(k r) = \frac{|N_{12}(k r)|^2}{|D_{12}(kr)|^2} \\
&= \frac{|\int_{0}^{\pi} \int_{0}^{2\pi} T_1(\phi,\theta,k r) T_2^*(\phi,\theta,k r) e^{-jk r cos\theta} sin\theta d\theta d\phi|^2}{|\sqrt{ \int_{0}^{\pi} \int_{0}^{2\pi} |T_1(\phi,\theta,k r)|^2 sin\theta d\theta d\phi } \sqrt{\int_{0}^{\pi} \int_{0}^{2\pi}|T_2(\phi,\theta,k r)|^2 sin\theta d\theta d\phi}|^2}.
\label{eq:MSCdir}
    \end{aligned}
\end{equation}


Moreover, the general expression of the directivity of a first-order differential microphone is given by the following relationship:

\begin{equation}
	\begin{aligned}
	T_i(\Omega_i) = \alpha_i + (1 - \alpha_i) \cos{\Omega_i},
	\end{aligned}
\end{equation}


where $i \in [1,2]$ is the microphone index, $\Omega_i$ is the angle between wave incidence and microphone orientation axis, and $\alpha_i \in [0,1]$ is the directivity parameter of the microphone $i$, which ranges from bidirectional ($\alpha_i = 0$) to omnidirectional ($\alpha_i = 1$). \\


For first-order differential microphones, there is a closed-form expression for the numerator and denominator of Eq.~(\ref{eq:MSCdir}):
\begin{equation}
	\begin{aligned}
    &N_{12}(k r) =  \frac{\alpha_1 \alpha_2 sin(kr)}{kr} \\
    &+ \frac{(1-\alpha_2)(1-\alpha_2)(x_1x_2+y_1y_2)}{(kr)^3}(sin(kr)-kr cos(kr)) \\
    &+ \frac{z_1 z_2}{kr^3}[ ( (kr)^2 sin(kr) + 2kr cos(kr) )(1-\alpha_1)(1-\alpha_2) 
    + 2 sin(kr)(1-\alpha_1)(1-\alpha_2) ] \\
    &+ \frac{z_1}{(kr)^3}[ j(kr)^2 \alpha_2 cos(kr)(\alpha_1-1) + jkr \alpha_2 sin(kr)(1+\alpha_1) ] \\
    &+ \frac{z_2}{(kr)^3}[ j(kr)^2 \alpha_1 cos(kr)(\alpha_2-1) + jkr \alpha_1 sin(kr)(1+\alpha_2) ],\\
    &D_{12}(kr) =  \frac{\sqrt{3 \alpha_1^2+(1-\alpha_1)^2}\sqrt{3 \alpha_2^2+(1-\alpha_2)^2}}{3},
    \label{eq:closedform_msc}
    \end{aligned}
\end{equation}
where $x_i$, $y_i$ and $z_i$ are the cartesian coordinates of the wave incidence angle $\Omega_i$. \todo{check}.





\section{Reverberation}

In the context of room acoustics, reverberation refers to \textit{"the energy of a sound source that reaches a listener indirectly, by reflecting from surfaces within the surrounding space occupied by the sound source and the listener"} \cite{begault20003}. 
Conversely, in anechoic or free-field conditions, where reverberation is not present, only the direct path of the sound source exists.
Assuming linearity and time-invariance, room reverberation can be fully characterised by its impulse response (IR). \\

Reverberation models often consider two differentiated parts of the reverberant tail, based on both physical and perceptual characteristics: the \textit{early reflections} and the \textit{late reverberation}. 
Early reflections, as the name suggests, refers to the individual sound paths arriving to the listener after a few reflections on the room surfaces, which cause some degree of attenuation. Early reflections typically arrive with a time difference between 1 and 80 ms after the direct path \cite{begault20003}. 
The term late reverberation encompasses all sound paths arriving to the listener after many reflections. Since the temporal density of such reflections increases with time, late reverberation is often modelled in statistical terms.
An schematic representation of a room impulse response (RIR) is shown in Figure~\ref{fig:rir}.\\

\begin{figure}[htbp]
	\begin{center}
	\includegraphics[width=\textwidth]{Figures/Introduction/Acoustic_room_impulse_response.jpg}
	\caption{Room impulse response model, from \cite{murphy2017acoustic}.}
	\label{fig:rir}
	\end{center}
\end{figure}

By following this model, a RIR $h(t)$ can be described as a sequential combination of responses:
\begin{equation}
	h(t) = h_D(t) + h_R(t),
\label{eq:directreverberant}
\end{equation}
where $h_D(t)$ and $h_R(t)$ represent the \textit{direct} (direct path plus early reflections) and \textit{reverberant} (late reverberation) components of the RIR. respectively.\\


The room impulse response is a function of both the source and the receiver locations. Different levels, delays and directions of direct path and early reflections can obtained from measurements in the same room. However, it is generally assumed that the late reverberation is fixed for a given room, regardless of source/receiver positions. \\

Room reverberation plays an important role in psychoacoustics. While early reflections are usually perceived together with the direct path as a single auditory event, due to the \textit{precedence effect} \cite{haas1972influence}, late reverberation has often an influence on the received signal. In the specific case of speech, late reverberation is associated with a loss of intelligibility \cite{braun2018speech}.
In the context of spatial perception, it has been shown that early reflections help the localization and externalization of sources [\todo{Improving Externalization in Ambisonic Binaural Decoding}], while the late reverberation is associated with a spaciousness perception of the room \cite{begault20003}. \\


There are a number of measurable parameters which help to characterise room acoustics. 
Perhaps one of the most widespread is the \textit{reverberation time} $T_{60}$ \cite{kuttruff2016room}. It represents the time required for the reverberant sound field power to decay by 60 dB.
Reverberation time can be accurately computed from the room geometry \cite{sabine1927collected} or from the IR \cite{schroeder1965new}.

In the latter case, the $T_{60}$ value is usually estimated from the \textit{Energy Decay Curve} (EDC), which is defined as:

\begin{equation}
	\text{EDC}(t) = 10 \log_{10} \sum_{t'=t}^{\infty} h^2(t),
\end{equation}

where $h(t)$ represents the room impulse response. The values are normalized such that the maximum peak of the curve corresponds to 0 dB.

The EDC is usually modelled as a straight line in logarithmic scale. Therefore, the $T_{60}$ estimation is performed by estimating the slope of a straight line between two reference levels on the EDC time series.
Some of the most used reference levels receive specific names: \textit{Early Decay Time} (EDT), $T_{60}$, and reverberation times $T_{10}$, $T_{20}$ and $T_{30}$. Table\ref{tab:reverberationtimes} shows their correspondent reference levels, where the maximum energy peak is normalized to 0 dB. An schematic representation of the reference levels is depicted in Figure~\ref{fig:reverberationtimes}.

An alternative parameter is the \textit{decay rate} $\alpha_{60}$, which is related to reverberation time $T_{60}$ as:
\begin{equation}
	\alpha_{60} = \frac{3 \ln{(10)}} {T_{60}} (\text{dB/s}).
\end{equation}
 
The decay rate is thus the slope of the EDC curve, in logarithmic scale, expressed in dB per second.  

To conclude, it is important to notice that reverberation time is frequency-dependent. Accordingly, it is usual to report it for octave or third-octave bands, or alternatively to provide its value at a specific frequency. \\


\begin{table}[t]
\caption{Reverberation time computation: usual reference levels}
\begin{center}
\begin{tabular}{ccccc}
\toprule
   & $\text{EDT}$ & $T_{10}$ & $T_{20}$ & $T_{30}$ \\
\midrule
$L_{max} (dB)$ & 0 & -5 & -5 & -5  \\
$L_{min} (dB)$ & -10 & -15 & -25 & -35 \\
\bottomrule
\end{tabular}
\label{tab:reverberationtimes}
\end{center}
\end{table}

\begin{figure}[htbp]
	\begin{center}
	\includegraphics[width=\textwidth]{Figures/Introduction/edt.png}
	\caption{Room impulse response model, adapted from \url{http://www.bnoack.com/}.}
	\label{fig:reverberationtimes}
	\end{center}
\end{figure}


The \textit{Direct to Reverberant Ratio} (DRR) is another relevant acoustic parameter. DDR represents the ratio between direct and reverberant parts of the RIR, as defined in Eq.~\ref{eq:directreverberant}:
\begin{equation}
	DRR = 10 \log_{10} \frac{ \sum_{t=1}^{L_D} h^2_D(t) }{ \sum_{t=1}^{L_R} h^2_R(t)},
\end{equation}
with $L_D$ and $L_R$ as the length of the direct $h_D(t)$ and reverberant $h_R(t)$ filters, respectively.  
At a psychoacoustic level, the direct to reverberant ratio is one of the main cues for distance perception \cite{begault20003}. 

Since the direct path and early reflections (but not the late reverberation) depend on the relative position between source and receiver , the filter $h_D(t)$ and therefore the DRR are as well location-dependent. For a given room, the source-receiver distance that produces a DRR of 0 dB is known as the \textit{critical distance}. 


\subsection{SOFA Conventions}

The availability of recorded room impulse responses is of great importance to many acoustic signal processing problems. 
As we have seen, many different RIRs can be obtained from the same room, by just varying the position of the source and the receiver; when the number of source and receiver positions increases, the total amount of measurements increases geometrically. 
Besides that, the actual format and organisation of the produced data (not only the RIR themselves, but also the source/position annotations) can be arbitrarily different when produced by different groups of people.

In order to overcome potential interoperatibility and reusability issues, the \textit{Spatially Oriented Format for Acoustics} (SOFA) convention \cite{majdak2013spatially}, also known as the \textit{AES-69} standard \cite{majdak2015aes69}, proposes a unified file format for the storage of IR-related data. 
Despite that SOFA was initially created with a focus on \textit{Head-Related Impulse Response} (HRIR) data, its structure is very convenient to any kind of multi-location impulse responses, including ambisonics.


\section{Signal Model}


Let us consider a sound source represented by the signal $s(t)$, located in a given acoustic enclosure characterised by its room impulse response $h(t)$. The resulting reverberant signal $x(t)$ can be therefore described as the \textit{convolutive mixture} of the source and the RIR: 
\begin{equation}
	x(t) = s(t) \ast h(t).
\label{eq:convolutivemixture}
\end{equation}

When dealing with multichannel room impulse responses, as it is the case in ambisonics, the multichannel reverberant signal $x_m(t)$ is obtained by the convolutive mixture of each RIR channel independently:
\begin{equation}
	x_m(t) = s(t) \ast h_m(t).
\label{eq:convolutivemixturemultichannel}
\end{equation}

The time domain convolution operation, under certain assumptions, is equivalent to the multiplication in frequency domain. By doing so, Eq.~\ref{eq:convolutivemixture} can be expressed as:
\begin{equation}
	X(k,n) = S(k,n) H(k,n).
\label{eq:multiplicativetransferfunction}
\end{equation}

Eq.~\ref{eq:multiplicativetransferfunction}, also known as the \textit{Multiplicative Transfer Function (MTF) model} is only valid when the length of the filter $h(t)$ is smaller than the length of the analysis window used in the STFT. 


On the contrary, when the filter $h(t)$ spans across several analysis windows, the resulting model is referred to as the \textit{Convolutive Transfer Function (CTF) model}:
\begin{equation}
	X(k, n) = \sum_{l=0}^{L_h-1} H(k, l) S(k, n-l), 
\label{eq:convolutivetransferfunction}
\end{equation} 
where $L_h$ is the length of the filter $H(k, n)$ in time frames. 



\todo{isotropic noise}



\section{Practical Considerations}
% TODO I don't remember what is this


\chapter{Blind reverberation time estimation}



% This is the modified v2 as after Emilia's comments

%%%%%%%%%%%%%%%%%%%%%%%%%%%%%%%%%%%%%%%%%%%%%%%%%%%%%%%%%%%%
%%%%%%%%%%%%%%%%%%%%%%%%%%%%%%%%%%%%%%%%%%%%%%%%%%%%%%%%%%%%
\section{Introduction}

%The knowledge about the acoustic properties of an enclosure results of the greatest interest for many situations within the microphone array and acoustic signal processing fields. 
Knowledge about the acoustic properties of an enclosure is a fundamental topic with many applications in the microphone array and acoustic signal processing field.
Problems such as dereverberation \cite{braun2018evaluation} or source separation \cite{gannot2017consolidated} may benefit from this information, 
%or might even require \textit{a priori} acoustic parameter estimations. 
and may require prior estimation of the related parameters.
%Reverberation time $T_{60}$ \cite{kuttruff2016room} might be one of the most widespread acoustic parameters; it represents the time required 
%%for a sound in the diffuse field to decrease its energy envelope by 60 dB. 
%for the reverberant sound field power to decay by 60 dB.
%Reverberation time can be accurately computed from the room geometry \cite{sabine1927collected} or from the Impulse Response (IR) \cite{schroeder1965new}; the problem of $T_{60}$ estimation just from observations of the reverberant signal itself is referred to as the blind reverberation time estimation, and it still remains an open research question.

The 2016 Acoustic Characterisation of Environments (ACE) Challenge \cite{eaton2016estimation} gathered dozens of methods designed for blind $T_{60}$ and Direct-to-Reverberation Ratio (DRR) estimation; nowadays, it is still considered as a state-of-the-art source for 
%method comparison, and the evaluation metrics used have become standard. 
%evaluating performance of BRTE and comparing different methods
performance evaluation and comparison among methods.

Most of the model-based $T_{60}$ estimation algorithms consider the reverberant signal envelope as an exponential decay, so that the problem is reduced to finding a signal offset and estimate the decay rate. 
%For instance, the best scoring $T_{60}$ estimation algorithm, in terms of the Pearson correlation, implements a variant of this idea on the narrowband STFT spectrum \cite{prego2012blind}; this method will be used in next sections as the baseline method. 
Moreover, in last years, data-driven models have outperformed the previous state-of-the-art results \cite{gamper2018blind, looney2020joint, bryan2020impulse}.
%Some representative examples of such approaches are based on multi-layer perceptrons \cite{xiong2018joint} or convolutional neural networks (CNN) \cite{gamper2018blind, looney2020joint, bryan2020impulse}.
A comparative review on single-channel blind $T_{60}$ estimation algorithms was recently published \cite{lollmann2019comparative}. 
%Apart from providing a complete state-of-the-art overview, this work extends the ACE Challenge evaluation to more recent methods.  

However, most of the existing reverberation time estimation methods focus on the single-channel case. A representative example can be drawn from the ACE Challenge, where, despite the fact that one of the reverberant datasets was recorded with an \textit{em32 Eigenmike} spherical microphone array, none of the methods use of it for the $T_{60}$ estimation task. 

%one of the data subsets was recorded with an \textit{em32 Eigenmike} spherical microphone array, featuring 32 capsules. However, only one of the participants \cite{chen2015estimation} made use of those multichannel recordings, but just for the DRR estimation problem. 
%In a slightly different context, the method proposed in \cite{falcon2019machine} estimates reverberation time from room geometry; while the groundtruth values are computed through spatial filtering of Eigenmike recordings, those are just employed for algorithm training. 

On the other hand, recent years have witnessed a growing interest in immersive audio for virtual and augmented reality.
This situation has consolidated Ambisonics \cite{zotter2019ambisonics} as the \textit{de facto} standard for spatial audio. Dedicated spherical microphone arrays have reached the market in last years; their multichannel nature makes possible spatial manipulations that 
%improve traditional signal processing methods.
complement traditional signal enhancement methods.

In this chapter, we present a novel approach to the problem of multichannel blind reverberation time estimation, specifically focusing on first order ambisonic (FOA) recordings. 
%The method is based on dereverberation plus IR estimation by system identification. 
The method is based on a dereverberation stage followed by system identification.
To the best of our knowledge, the proposed algorithm is the first reverberation time estimation method specifically designed for first order ambisonic audio. 

The rest of the chapter is organized as follows. Section~\ref{sec:signalmodel} introduces the nomenclature and the signal model. Sections~\ref{sec:baseline} and \ref{sec:proposed} describe the baseline and the proposed methods, respectively. The experimental setup is described in Section~\ref{sec:experimental}, and the results are discussed in Section~\ref{sec:results}. Finally, a conclusion is presented in Section~\ref{sec:conclusion}.

%%%%%%%%%%%%%%%%%%%%%%%%%%%%%%%%%%%%%%%%%%%%%%%%%%%%%%%%%%%%
%%%%%%%%%%%%%%%%%%%%%%%%%%%%%%%%%%%%%%%%%%%%%%%%%%%%%%%%%%%%
\section{Signal Model}
\label{sec:signalmodel}

Let us consider a FOA signal $x_n^m(t)$, with $M=4$ as the number of channels.
Let us further assume the convolutive mixture signal model described in Eq.~\ref{eq:convolutivemixturemultichannel}, where the reverberant signal $x_n^m(t)$ represents the signal captured by an ideal spherical microphone array located in a reverberant enclosure. 
Let $s(t)$ denote the signal of the only sound source present in the scene, and $h_n^m(t)$ denote the ambisonic RIR modelling the acoustic enclosure:
\begin{equation}
	x_n^m(t) = s(t) \ast h_n^m(t)
\end{equation}

It is important to remark that $T_{60}$ estimation here assumes no receiver directionality. In an ambisonic context, this corresponds to the zeroth order component. Therefore, in what follows, all methods estimating IR parameters will be applied to the zeroth order channel, $x_0(t)$.

%%%%%%%%%%%%%%%%%%%%%%%%%%%%%%%%%%%%%%%%%%%%%%%%%%%%%%%%%%%%
%%%%%%%%%%%%%%%%%%%%%%%%%%%%%%%%%%%%%%%%%%%%%%%%%%%%%%%%%%%%
\section{Baseline method}
\label{sec:baseline}


The baseline algorithm, taken from \cite{prego2012blind}, is based on the detection of abrupt event offsets in the time-frequency domain. The subband energy decay on the transitions can be then used to compute an estimate of the full-band decay. This method performed best in the ACE Challenge regarding the Pearson correlation coefficient between estimated and true $T_{60}$\cite{eaton2016estimation}.

Let us consider the zeroth order channel of the recorded signal, $x_0(t)$, and its Short-Time Frequency Transform (STFT) counterpart $X_0(k,n)$.
%Although the method could be potentially applied to any of the channels, in physical terms it makes more sense to work with an omnidirectional sound field representation; hence the usage of the zeroth order channel.
The \textit{subband energy} $\bar{E}(k,n)$ of the recorded signal can be expressed as:\begin{equation}
	\bar{E}(k,n) = |X_0(k,n)|^2.
\end{equation}

A \textit{Free Decay Region} (FDR) is defined as a group of consecutive bins within the same subband which exhibit a monotonically decreasing energy. 
A FDR search is performed on the subband energy spectrogram $\bar{E}(k,n)$: for each band, the algorithm tries to find at least one FDR, iterartively reducing the FDR length if no candidates are found. 

The next step is the estimation of the reverberation time, which is performed using a subband equivalent of Schroeder's method \cite{schroeder1965new}.
The \textit{Subband Energy Decay Function} (SEDF) associated with a given FDR is computed as:

\begin{equation}
	\bar{c}(k,n) = 10 \log_{10} \frac{\sum_{\nu=n}^{L_c-1} \bar{E}(k,\nu)} {\sum_{\nu=0}^{L_c-1} \bar{E}(k,\nu)} \text{dB},
\end{equation}
where $n = 0 \ldots, L_c-1$ spans the length of the FDR. A linear regression is then performed on each SEDF curve: $T_{60}$ is computed as the time required by the resulting line to reach the $-60 \text{ dB}$ reference.

This procedure yields a $T_{60}$ estimate per FDR. In order to obtain a global estimate, the algorithm proposes a two-step statistical filtering. First, it obtains a narrowband estimate as the median of all estimates within each subband. Then, the resulting broadband value $\bar{T}_{60}$ is computed as the median of all subband estimates.
The last step of the method is the expansion of the resulting dynamic range by a linear mapping. This procedure is required because of the compression introduced by the median operator. The final value $T_{60}$ is thus a linear mapping of $\bar{T}_{60}$, where the parameters $\alpha$ and $\beta$ might be obtained by linear regression on a training stage:
\begin{equation}
	T_{60} = \alpha \bar{T}_{60} + \beta
\label{eq:expansion}
	\end{equation}



%%%%%%%%%%%%%%%%%%%%%%%%%%%%%%%%%%%%%%%%%%%%%%%%%%%%%%%%%%%%
%%%%%%%%%%%%%%%%%%%%%%%%%%%%%%%%%%%%%%%%%%%%%%%%%%%%%%%%%%%%
\section{Proposed method}
\label{sec:proposed}

We propose a novel method for reverberation time estimation, based on two steps: signal dereverberation, and system identification. The main idea consist in obtaining an estimate of the dereverberated signal, which is later used for estimating the multichannel IR given the recorded reverberant signal. The reverberation time can be thus computed by the decay slope of the estimated IR. 

\subsection{Dereverberation}

Let us consider now the \textit{CTF} model (Eq.~\ref{eq:convolutivetransferfunction}) version of the proposed signal model:
\begin{equation}
\label{eq:ctf}
	X_m(k, n) = \sum_{l=0}^{L_h-1} H_m(k, l) S(k, n-l), 
\end{equation} 
where the multichannel filter $H_m(k, l)$ of length $L_h$ contains the \textit{CTF coefficients} between the source and the microphones.

Considering the room impulse response model of Eq.~\ref{eq:directreverberant}, 
it is possible to sequentially split the former expression in the following way:
\begin{equation}
\label{eq:model}
\begin{aligned}
	&X_m(k, n) = D_m(k, n) + R_m(k,n) = \\
	=\sum_{l=0}^{\tau-1} H_m(&k, l) S(k, n-l) + \sum_{l=\tau}^{L_h-1} H_m(k, l) S(k, n-l),
\end{aligned}
\end{equation} 
%\begin{subequations}
%\begin{equation}
%\label{eq:model}
%	X_m(k, n) = D_m(k, n) + R_m(k, n),
%\end{equation} 
%\begin{equation}
%	D_m(k, n) = \sum_{l=0}^{\tau-1} H_m(k, l) S(k, n-l),
%\end{equation} 
%\begin{equation}
%\begin{aligned}
%	R_m(k_n) = \sum_{l=\tau}^{L_h-1} H_m(k, l) S(k, n-l),
%\end{aligned}
%\end{equation} 
%\end{subequations}
where the parameter $\tau$ represents the \textit{mixing time}, which states the transition time between early reflections and late reverberation. 
In other words, the captured signal is divided between a \textit{direct} part $D_m(k, n)$, containing the direct path and the early reflections, and a \textit{reverberant} part $R_m(k, n)$, which mainly contains the diffuse part of the reverberation.

Assuming a Multichannel Auto-Regressive (MAR) model, $R_m(k, n)$ can be expressed as a multichannel Infinite Impulse Response (IIR) filter applied to the recorded signal:
\begin{equation} 
\label{eq:mar}
	R_m(k, n) = \sum_{i=1}^{M} \sum_{l=0}^{L_g-1} X_i(k,n-\tau-l) G_{mi}(k, l),
\end{equation}
where the coefficients $G_{mi}(k,l) \in \mathbb{C}$  model the relation between channels $m$ and $i$, and have a length of $L_g$ frames. 

By grouping all time frames $n = 1 \ldots, N-1$, it is possible to express Eq.~\ref{eq:mar} in vector notation:
\begin{subequations} 
\begin{equation} 
	\bm{R}_m(k) =  \tilde{\bm{X}}_{\tau}(k) \bm{G}_{m}(k),
\end{equation}
\begin{equation} 
	\tilde{\bm{X}}_{\tau}(k) = [ \tilde{\bm{X}}_{\tau,1}(k), \ldots, \tilde{\bm{X}}_{\tau,M}(k)],
\end{equation}
\end{subequations}
where $\tilde{\bm{X}}_{\tau,m}(k)$ is a $N \times L_g$ matrix, and $\bm{R}_m(k)$ and $\bm{G}_{m}(k)$ are column vectors with lengths $N$ and $L_gM$, respectively.

Finally, the expression can be further simplified by omitting the frequency dependence, and by expressing the channels as columns in the vector notation. Substituting this expression in Eq.~\ref{eq:model} leads to the MAR equation:
\begin{equation} 
\label{eq:D}
	\bm{D} = \bm{X} - \tilde{\bm{X}}_{\tau} \bm{G}.
\end{equation}

Here, the dereverberation problem consists in the estimation of the MIMO filter $\bm{G}$, so that the \textit{clean} signal $\bm{D}$ (containing both direct path and early reflections) can be computed.

The algorithm proposed here is based on the method described in \cite{jukic2015group}. In this case, the dereverberation problem is tackled as an optimization problem, considering that the spectrograms of the reverberant signal are less sparse than those of the corresponding \textit{clean}, and ensuring that the inter-channel signal properties are mantained.
%(hence the reference to the \textit{group} sparsity). 
Although the presented method is applied on the whole signal in \textit{batch} mode, alternative \textit{online} methods could be also used, e.g. \cite{braun2016online}.
%Given that this is a \textit{batch} processing algorithm, the whole audio clip is processed at once. Alternative solutions using online MAR models could be also used if required \cite{braun2016online}.

By using \textit{iteratively reweighted least squares} (IRSL) \cite{chartrand2008iteratively}, it can be shown that an iterative solution for the estimation of $\bm{G}$ at the iteration $(i)$ is given by the following expression:

\begin{equation}
\begin{aligned}
	\label{eq:G}
	\bm{G}^{(i)} = ( \tilde{\bm{X}}_{\tau}^H \bm{W}^{(i)} \tilde{\bm{X}}_{\tau} )^{-1} \tilde{\bm{X}}_{\tau}^H \bm{W}^{(i)} \bm{X}, 
\end{aligned}
\end{equation}
 where $\bm{W}^{(i)}$ is a $N \times N$ diagonal matrix whose diagonal values, $w_n^{(i)}$, can be updated as:
\begin{equation}
\begin{aligned}
\label{eq:w}
	w_n^{(i)} = ( \bm{d}_n^{H (i-1)} \bm{\Phi}^{-1 (i-1)} \bm{d}_n^{(i-1)}  )^{ \frac{p-2}{2} } + \epsilon.
\end{aligned}
\end{equation}

In turn, $\bm{d}_n$ represents the rows of $\bm{D}$ arranged as column vectors of length $M$, $\bm{\Phi}$ is the $M \times M$ Spatial Covariance Matrix (SCM) of $\bm{D}$, $\epsilon$ is an arbitrary small positive value, and $p \leq 1$. The computation and update of the SCM matrix is given by:
\begin{equation}
\begin{aligned}
\label{eq:phi}
	\bm{\Phi}^{(i)} = \frac{1}{N} \bm{D}^{T(i)} \bm{W}^{(i)} \bm{D}^{*(i)}.
\end{aligned}
\end{equation}


%It is interesting to notice that, in the original proposal, the SCM matrix is assigned by default to the identity matrix, and its actual estimation is regarded as optional. However, it is known that the SCM matrix is only equal to the identity matrix in the spherical harmonic domain under isotropic noise \cite{epain2016spherical}. Furthermore, the method has also reported a slightly better performance with the SCM estimation.

To conclude the dereverberation method, Eqs.~\ref{eq:D}, \ref{eq:G}, \ref{eq:w} and \ref{eq:phi} can be applied iteratively, starting by updating Eq.~\ref{eq:w}, until convergence is reached: 
\begin{equation}
 \| \bm{D}^{(i)} - \bm{D}^{(i-1)} \|_F / \| \bm{D}^{(i)} \|_F < \eta, 
\end{equation}
where $\eta$ is an arbitrary small positive value, or alternatively until the maximum number of iterations $i_{max}$ is exceeded. For the initialization, the following values are proposed: $\bm{D} = \bm{X}$ and $\bm{\Phi} = \bm{I}_M$ (the identity matrix of size $M \times M$).\\

%%%%%%%%%%%%%%%%%%%%%%%%%%%%%%%%%%%%%%%%%%%%%%%%%%%%%%%%%%%%
\subsection{System Identification}

%The output of the dereverberation step is the multichannel signal $D_m$, which ideally contains the direct plus early reflection components of the source. 
%Therefore, given the reverberant signal $X_m$ and the dereverberated signal $D_m$, an estimate of the late room impulse response might be derived by using System Identification (SID) through the pseudo-inverse method. 
%As in Section~\ref{sec:baseline}, we are primarily interested on the response of the omnidirectional channel; for that reason, the filter estimation is performed with the zeroth order components of both recorded and dereverberated signals: 
%%\begin{equation}
%%	\hat{\bm{H}}_0 = ( \pmb{D}_0^T \pmb{D}_0 )^{-1} \pmb{D}_0^T \bm{X}_0
%%\end{equation}
%%\begin{equation}
%%	\hat{H}_0(k,n) = ( D_0(k,n)^T D_0(k,n) )^{-1} D_0(k,n)^T X_0(k,n)
%%\end{equation}
%\begin{equation}
%	\hat{H}_0 = ( D_0^T D_0 )^{-1} D_0^T X_0
%\end{equation}

The output of the dereverberation step is the multichannel signal $D_m$, which ideally contains the direct plus early reflection components of the source. Therefore, given the reverberant signal $X_m$ and the dereverberated signal $D_m$, an estimate of the late room impulse response might be derived by identifying the filter connecting the two. 
As stated in Section~\ref{sec:signalmodel}, we are primarily interested on the response of the omnidirectional channel; for that reason, the filter estimation is performed with the zeroth order components of both recorded  and dereverberated signals. We perform system identification directly in the STFT through a linear fit between input and output independently for every frequency bin:
\begin{equation}
   \hat{H}_0(k) = \frac{\mathbf{d}^{\mathrm{H}}_0(k) \mathbf{x}^{}_0(k) } {\mathbf{d}^{\mathrm{H}}_0(k) \mathbf{d}^{}_0(k)},
\end{equation}
where $\mathbf{d}_0, \mathbf{x}_0$ are $N\times 1$ length vectors. To avoid complex cross-band modeling of the system response, we use a long STFT window, assumed longer than the twice the length of the IR so that a reduction of the CTF to a Multiplicative Transfer Function (MTF) holds \cite{avargel2007multiplicative}.

As a last sep, the estimated time-frequency filter $\hat{H}_0(k,n)$ is transformed into the time domain filter $\hat{h}(t)$. The $T_{60}$ is then computed by linear fitting of the Schroeder integral 
in the $[-5, -15] $ dB range ($T_{10}$ estimation method), after filtering $\hat{h}(t)$ with an octave-band filter centered at 1 kHz. 

%It is important to notice that the first order components of the ambisonic sound field are not used for the system identification step, in a similar manner as described in Section~\ref{sec:baseline}. However, they do play an important role on the dereverberation step; as shown in \cite{jukic2015group}, the number of channels used for the MIMO IIR filter is directly proportional to the dereverberation performance.  



%%%%%%%%%%%%%%%%%%%%%%%%%%%%%%%%%%%%%%%%%%%%%%%%%%%%%%%%%%%%
%%%%%%%%%%%%%%%%%%%%%%%%%%%%%%%%%%%%%%%%%%%%%%%%%%%%%%%%%%%%
\section{Experimental setup}
\label{sec:experimental}

\vspace{-2mm}
\subsection{Dataset}

The proposed method is evaluated using two different reverberant datasets, containing recordings of \textit{speech} and \textit{drums} respectively. 
In order to have full control over the reverberation conditions in the experimental setup, the audio clips under consideration have been rendered by the convolutive mixture of clean monophonic recordings with FOA IRs.

The \textit{speech} dataset is composed of the LibriSpeech \cite{panayotov2015librispeech} \textit{test-clean} audio samples longer than 25 s, making a total of 30 audio clips. It contains English language sentences by male and female speakers, often with a small level of background noise. We have used only a 20 s long excerpt of each clip, preceded by an initial offset of 5 s. 
The \textit{drums} dataset is the \textit{test} subset of the isolated drum recordings from the DSD100 dataset \cite{SiSEC16}. It contains 50 different audio clips, covering a wide range of music and mixing styles. The same audio lengths and offsets as in the previous case are applied.

The IRs are FOA room impulse responses simulated by the image method with the \textit{Multichannel Acoustic Signal Processing} library \cite{masp}. There are 9 different IRs of 1 s, with random $T_{60}$ values in the range between 0.4 s and 1.1 s approximately, estimated by the $T_{10}$ method at the 1 kHz band. The angular position of the sources is randomized for each IR, while the receiver position is fixed at the room center, which has a size of $10.2 \times 7.1 \times 3.2$ m. The source distance is set to half the \textit{critical distance}, thus providing positive DRRs. 

The combination of the dry audio clips with the IRs yields a total of 270 and 450 audio clips for the \textit{speech} and \textit{drums} datasets, respectively, after removing the audio clips which mostly contain silence. Those datasets will be referred in the following as the \textit{evaluation} datasets. 

Finally, the baseline method requires a previous \textit{fitting} step for the computation of the mapping parameters $\alpha$ and $\beta$ from Eq.~\ref{eq:expansion}. The procedure has been performed as follows.
For the \textit{speech} dataset, we selected again the subset of audio clips longer than 25 s, but in this case on the \textit{dev-clean} dataset, which yields a total of 20 audio clips. 
For the \textit{drums} dataset, we used the 50 clips of the \textit{development} subset.
The generation of the convolutive mixes has followed the same procedure as in the previous case. We will refer to the resulting datasets as the \textit{development} datasets. 

\subsection{Setup}
\label{sec:setup}


The sampling frequency for all methods is 8 kHz.
For the baseline system, the window size is 1024 samples long, with an overlap of 256 samples. The FDR length is set to 500 ms, which has been reported as the ideal theoretical minimum \cite{prego2012blind}; it corresponds to a FDR length of $L_c = 15$ samples. 
At any frequency band, the value of $L_c$ is iteratively decreased if no FDR is found, until a minimum value of 3 samples (96 ms). If still no FDR is found, the sound clip is discarded. 

In order to compute $\alpha$ and $\beta$, we run the baseline method on both \textit{development} datasets. For each IR, the mean and standard deviation of the results are computed across all sound clips. Then, these values are used for a \textit{weighted least squares} linear regression against the true $T_{60}$ values.
The results are shown in Table~\ref{tab:fitting}, where $\sigma$ represents the joint standard deviation of $\alpha$ and $\beta$ after the linear regression; 
the resulting values are in the same range as the values reported in \cite{prego2012blind}. 


\begin{table}[t]
\caption{Baseline system: linear regression parameters}
\begin{center}
\begin{tabular}{cccc}
\toprule
Dataset & $\alpha$  & $\beta$  & $\sigma$ \\
\midrule
Speech & 6.6619 & -1.4517 & 0.2131  \\
Drums  & 8.2421 & -2.1939 & 1.0055  \\
\bottomrule
\end{tabular}
\label{tab:fitting}
\end{center}
\end{table}

In the dereverberation stage, the STFT uses a small window size of 128 samples, with 64 samples overlap. The value of $p$ is set to 0.25, given the good results reported in \cite{jukic2015group}. Other parameter values are $\tau=2$, $i_{max} = 10$, $\eta=10^{-4}$ and  $\epsilon=10^{-4}$. After an exploratory search, the length of the IIR filter $L_g = 20$ has been chosen as a compromise between method performance and computation time. 
We have observed a tendency towards poor dereverberation and non-convergence of the IRSL when using small values of $L_g$ and short audios.

For the SID, the recorded and dereverberated signals are reshaped into much larger STFTs, with a window size of 8 s and a hop size of 0.5 s. The predicted filter size is 1 s. 

For both \textit{evaluation} datasets, the two presented methods are employed; we will refer to them as \textit{Baseline} and \textit{MAR+SID}. Furthermore, with the aim of evaluating the performance of the SID method in an isolated manner, we have included a third method, \textit{Oracle SID}. As its name suggests, it performs the System Identification step using the true anechoic signal.
%, as an ideal case of perfect dereverberation. 

\subsection{Evaluation metrics}
%Regarding method evaluation, 
We have considered the three metrics from the ACE Challenge \cite{eaton2016estimation}, all of them based on the differece between estimated and true values: the \textit{bias}, or mean error; the Mean Squared Error (\textit{MSE}); and the Pearson correlation coefficient. The evaluation has been performed after discarding the outliers, defined as the reverberation time estimates greater than 1.5 s.


%%%%%%%%%%%%%%%%%%%%%%%%%%%%%%%%%%%%%%%%%%%%%%%%%%%%%%%%%%%%
%%%%%%%%%%%%%%%%%%%%%%%%%%%%%%%%%%%%%%%%%%%%%%%%%%%%%%%%%%%%
\section{Results}
\label{sec:results}

\begin{table}[t]
\caption{Experiment results}
\begin{center}
\begin{tabular}{ccccc}
\toprule
 & \multicolumn{2}{c}{speech} & \multicolumn{2}{c}{drums} \\
%\cline{2-5} 
Metric & \textit{Baseline}&\textit{MAR+SID}&\textit{Baseline}&\textit{MAR+SID}\\
\midrule
Bias  & -0.0599    & \textbf{0.0305}   & \textbf{0.1521}     & 0.2568   \\
MSE   & 0.6366     & \textbf{0.0594}   & \textbf{13.9376}    & 16.5261  \\
$\rho$ & 0.8212    &\textbf{ 0.9848}   & 0.3705    			 & \textbf{0.7552}  \\ 
\hline
\end{tabular}
\label{tab:results}
\end{center}
\end{table}


Figure~\ref{fig:results1} shows the experiment result specified for all audio clips individually. Each boxplot represents the statistics of the mean estimation error (\textit{bias}) for a single audio clip subject to all 9 different IRs. The results are organized by method (rows) and dataset (columns). 
Figure~\ref{fig:results2} aggregates all experiment results into the same plot, showing the statistical distribution of the \textit{bias} per method and dataset. In this case, the \textit{Oracle SID} results are omitted for clarity.
The evaluation metrics for all methods are shown in Table~\ref{tab:results}.\\

According to the results, the proposed method clearly outperforms the baseline in the \textit{speech} dataset by a tenfold MSE improvement. For the \textit{drums} dataset, our method only outperforms the baseline regarding correlation. 
Nevertheless, an inspection of the statistical distribution of mean estimation errors in Figure~\ref{fig:results2} brings in an interesting observation: 
%the results of the presented method are much more concentrated around zero. In other words, 
the variability of the results given by our method is substantially smaller than the results of the baseline system. This behaviour is consistent across datasets: the mean error distributions with the \textit{speech} dataset are approximately five times narrower than with the \textit{drums} dataset, regardless of the method. \\
%The wide distribution of results with the baseline method might be a side effect of the dynamic range expansion of Eq.~\ref{eq:expansion}.

Moreover, all methods behave significantly better on the \textit{speech} dataset. The main reason might be the heterogeneity of the \textit{drums} dataset dataset with respect to dynamic range or timbre, and the potential application of audio effects of any kind.
Furthermore, some audio clips of the \textit{drums} dataset contain sounds with a high degree of self-similarity, such as cymbal rolls or exaggerated reverbs; these characteristics would explain the outliers on the proposed method results.
It is also interesting to notice the robustness of the proposed method against noise, present in the \textit{speech} dataset. 
%Although the signal model does not contemplate noise explicitly, such robustness is consistent with the behaviour described in \cite{jukic2015group}.
Such robustness is consistent with the behavior reported in \cite{jukic2015group}.\\

The performance of the \textit{ORACLE SID} method is close to ideal. The \textit{bias} is in all cases under 0.05 s (excepting a \textit{drums} clip containing mostly silence).
This result validates the system identification, and allows, in practical terms, a direct evaluation of the proposed method against the groundtruth values.

The results obtained in our analysis are very similar to the results reported in recent deep-learning state-of-the-art proposals, e.g. \cite{gamper2018blind}. 
Since all those methods perform single-channel estimation, and our method requieres FOA recordings, the results are not directly comparable. However, given the similar results obtained with the same evaluation metrics, it might be anticipated that out method may perform as well as other recent data-driven algorithms.
%Although a direct comparison is not possible due to the difference on the datasets under consideration, it might be anticipated that out method may perform as well as other recent data-driven algorithms.


%\begin{figure*}[htbp]
%	\begin{minipage}[b]{1.0\linewidth}
%		\centerline{\includegraphics[width=\textwidth]{Figures/ReverberationTimeEstimation/96.png}}
%		\centerline{(a) Estimation error computed for each audio clip.}\medskip
%	\end{minipage}
%	\begin{minipage}[b]{1.0\linewidth}
%		\centerline{\includegraphics[width=\textwidth]{Figures/ReverberationTimeEstimation/94.png}}
%		\centerline{(b) Total estimation error across audio clips and acoustic conditions. Top: boxplot. Bottom: histogram and density plot.}\medskip
%	\end{minipage}
%	\caption{Experiment results for \textit{speech} (left column) and \textit{drums} (right column) datasets.}
%	\label{fig:results}
%\end{figure*}

\begin{figure*}[htbp]
	\centerline{\includegraphics[width=1.3\textwidth]{Figures/ReverberationTimeEstimation/96.png}}
	\caption{Experiment results for \textit{speech} (left column) and \textit{drums} (right column) datasets. Estimation error computed for each audio clip.}
	\label{fig:results1}
\end{figure*}
\begin{figure*}[htbp]
	\centerline{\includegraphics[width=1.2\textwidth]{Figures/ReverberationTimeEstimation/94.png}}
	\caption{Experiment results for \textit{speech} (left column) and \textit{drums} (right column) datasets. Total estimation error across audio clips and acoustic conditions. Top: boxplot. Bottom: histogram and density plot}
	\label{fig:results2}
\end{figure*}




%%%%%%%%%%%%%%%%%%%%%%%%%%%%%%%%%%%%%%%%%%%%%%%%%%%%%%%%%%%%
%%%%%%%%%%%%%%%%%%%%%%%%%%%%%%%%%%%%%%%%%%%%%%%%%%%%%%%%%%%%
\section{Conclusion}
\label{sec:conclusion}

We have presented in this work a novel method for blind reverberation time estimation for multichannel audio, with the aim of applying it to the context of ambisonic recordings. Our method is based on a first dereverberation step,  performed by a multichannel autoregressive model of the late reverberation. The resulting dry signal is then used to estimate the impulse response decay by means of system identification. 
The performance of the method is evaluated in a simulated experimental environment with two different reverberant datasets, and compared against a state-of-the-art method. 
Results show that our method outperforms the baseline method in a majority of evaluation metrics and conditions, and consistently provides results with less variability than the baseline method. 
In future work, we plan to extend the experimental setup by using recorded IRs. Furthermore, the proposed method could be extended to the case of moving sources by using an \textit{online} autoregressive model. 
%Finally, an evaluation of the method in terms of the number of channels or the ambisonic order remains to be done.
Finally, an extension of the method for higher ambisonic orders remains to be done.

\chapter{Coherence Estimation}



\section{Introduction}

A number of practical applications benefit of the knowledge about the diffuseness of a sound field, including speech enhancement and dereverberation \cite{p_habets_dual-microphone_2006}, noise suppression \cite{ito_designing_2010}, source separation \cite{duong_under-determined_2009} or background estimation \cite{stefanakis_foreground_2015}. In the field of spatial audio, diffuseness estimation is often used for parametrization \cite{pulkki_directional_2006, politis_compass_2018}, Direction-of-Arrival estimation \cite{thiergart_localization_2009} or source separation \cite{motlicek_real-time_2013}.

\todo{In this paper}, we study diffuseness estimation by subjecting a tetrahedral microphone array to spherically isotropic noise fields.
The motivation for this work is, first, that tetrahedral arrays are a well known type of microphone arrays, which have today become popular for applications related to Virtual and Augmented Reality. 
Second, the spherical isotropic sound field is known to be a good approximation to the reverberant part of the sound field in a room \cite{elko_spatial_2001, mccowan_microphone_2003}, and therefore it would be interesting to investigate how different microphone arrays behave under such conditions.




\subsection{\label{subsec:3:1} Coherence analysis}

Diffuseness is commonly estimated through the
\textit{Magnitude Squared Coherence} (MSC) \cite{elko_spatial_2001} between two frequency-domain signals $S_1$ and $S_2$, as a function of the
\textit{wavenumber} $k$ and the microphone distance $r$:
\begin{equation}
    \text{MSC}_{12}(k r) =
	\frac{|\left\langle S_1(k r) S_2(k r)^* \right\rangle|^2}
	{\left\langle|S_1(k r)|^2\right\rangle \left\langle|S_2(k r)|^2\right\rangle},
    \label{eq:MSC}
\end{equation}
where the $\left\langle \cdot \right\rangle$ operator represents the temporal
expected value, and $^*$ defines the complex conjugate operator. In the case of spherical isotropic noise fields, Eq.~(\ref{eq:MSC})
can be expressed in terms of microphone directivity patterns
$T(\phi,\theta,k r)$ as \cite{elko_spatial_2001}:

\begin{equation}
	\begin{aligned}
&\text{MSC}_{12}(k r) = \frac{|N_{12}(k r)|^2}{|D_{12}(kr)|^2} \\
&= \frac{|\int_{0}^{\pi} \int_{0}^{2\pi} T_1(\phi,\theta,k r) T_2^*(\phi,\theta,k r) e^{-jk r cos\theta} sin\theta d\theta d\phi|^2}{|\sqrt{ \int_{0}^{\pi} \int_{0}^{2\pi} |T_1(\phi,\theta,k r)|^2 sin\theta d\theta d\phi } \sqrt{\int_{0}^{\pi} \int_{0}^{2\pi}|T_2(\phi,\theta,k r)|^2 sin\theta d\theta d\phi}|^2}.
\label{eq:MSCdir}
    \end{aligned}
\end{equation}


Moreover, the general expression of the directivity of a first-order differential microphone is given by the following relationship:

\begin{equation}
	\begin{aligned}
	T_i(\psi_i) = \alpha_i + (1 - \alpha_i) \cos{\psi_i},
	\end{aligned}
\end{equation}


where $i \in [1,2]$ is the microphone index, $\psi_i$ is the angle between wave incidence and microphone orientation axis, and $\alpha_i \in [0,1]$ is the directivity parameter of the microphone $i$, which ranges from bidirectional ($\alpha_i = 0$) to omnidirectional ($\alpha_i = 1$). \\


For first-order differential microphones, there is a closed-form expression for the numerator and denominator of Eq.~(\ref{eq:MSCdir}):
\begin{equation}
	\begin{aligned}
    &N_{12}(k r) =  \frac{\alpha_1 \alpha_2 sin(kr)}{kr} 
    + \frac{(1-\alpha_2)(1-\alpha_2)(x_1x_2+y_1y_2)}{(kr)^3}(sin(kr)-kr cos(kr)) \\
    &+ \frac{z_1 z_2}{kr^3}[ ( (kr)^2 sin(kr) + 2kr cos(kr) )(1-\alpha_1)(1-\alpha_2) 
    + 2 sin(kr)(1-\alpha_1)(1-\alpha_2) ] \\
    &+ \frac{z_1}{(kr)^3}[ j(kr)^2 \alpha_2 cos(kr)(\alpha_1-1) + jkr \alpha_2 sin(kr)(1+\alpha_1) ] \\
    &+ \frac{z_2}{(kr)^3}[ j(kr)^2 \alpha_1 cos(kr)(\alpha_2-1) + jkr \alpha_1 sin(kr)(1+\alpha_2) ],\\
    &D_{12}(kr) =  \frac{\sqrt{3 \alpha_1^2+(1-\alpha_1)^2}\sqrt{3 \alpha_2^2+(1-\alpha_2)^2}}{3},
    \label{eq:closedform_msc}
    \end{aligned}
\end{equation}
where: $ x_i = cos(\phi_i) sin(\theta_i); y_i = sin(\phi_i) sin(\theta_i); z_i = cos(\theta_i)$
refers to the wave incidence angle $\psi_i$ expressed in spherical coordinates (with azimuth $\phi$ and inclination $\theta$).



\subsection{\label{subsec:3:2}Diffuseness estimation in ambisonics}
Let us consider a sound field captured with a spherical microphone array,
which contains $Q$ microphones distributed around a spherical surface of radius $R$
at the angular positions given by the azimuth-inclination pairs
$\Omega_q = (\phi_q,\theta_q)$.
The captured frequency-domain signals $ X_q(k)$ can be represented as the spherical harmonic domain signals $X_{mn}(k)$ through the spherical harmonic transform of order $L$ \cite{bertet_3d_2006}:
\begin{equation}
        X_{mn}(k) = \sum^{Q} X_q(k) Y_{mn}(\Omega_q) \Gamma_m(kR),
    \label{eq:ambisonics_encoding}
\end{equation}
where $Y_{mn}(\Omega_q)$ are the \textit{real-valued spherical harmonics}, and $\Gamma_m(kR)$ are the
\textit{radial filters} or equalization terms of order $m$, with $m \in [0,L]$ and $n \in [-m,m]$.\\


Due to a number of practical reasons, it is desirable to distribute the microphone capsules in a uniform manner along the sphere, with the the regular tetrahedron being the simplest possible configuration
 \cite{gerzon_design_1975}. Capsule signals recorded with such topology receive the name of \textit{A-Format} signals. Conversely, the term \textit{B-Format} (\textit{ambisonics}) describes the application of Eq.~(\ref{eq:ambisonics_encoding}) (\textit{ambisonic encoding}) to the \textit{A-Format} signals.
One of the most common coherence estimators for first-order ambisonic frequency-domain signals $X_{mn}(k)$ is the \textit{diffuseness} $\Psi$ as defined in \textit{DirAC} \cite{pulkki_directional_2006}:



\begin{equation}
    \Psi(k) = 1 - \frac{2 ||\left\langle \mathbb{R}\{\bm{X_1}(k)X_0(k)^* \} \right\rangle|| }{ \left\langle||\bm{X_1}(k)||^2 + |X_0(k)|^2\right\rangle},
    \label{eq:psi}
\end{equation}
where $X_0(k) = X_{00}(k)$ 
and $\bm{X}_1(k) = [X_{1-1}(k), X_{10}(k), X_{11}(k)]^\intercal$ are \textit{SN3D}-normalized.
For the sake of clarity, we will further define the \textit{B-Format coherence} estimator $\Delta$ as:
\begin{equation}
	\Delta(k) = 1 - \Psi(k).
	\label{eq:delta}
\end{equation}


Under spherical isotropic noise, the theoretical coherence between any pair of zeroth and first order ambisonic virtual microphones is equal to 0 for all frequencies, due to orthogonality and symmetry of the spherical harmonics \cite{elko_spatial_2001}. . This result can be also assessed by Eq.~(\ref{eq:closedform_msc}).

However, there are several practical factors that might corrupt the coherence estimation, such as the approximation of the temporal expectation by time averaging \cite{thiergart_diffuseness_2011} in Eq.~(\ref{eq:psi}), or the non-ideal implementation of the radial filters $\Gamma_m(kR)$ \cite{schorkhuber_ambisonic_2017}.
In the following sections, we present several experiments that illustrate the behavior of different coherence estimators applied on the signals captured with a tetrahedral microphone subjected to spherical isotropic noise, using both simulated and real sound recordings.



\section{Methods}

\subsection{Simulation}
Spherical isotropic noise has been generated following the \textit{geometrical method} \cite{habets_generating_2007, habets_comments_2010}, using $N = 1024$ plane waves. The resulting \textit{A-Format} signals correspond to a virtual tetrahedral microphone array mimicking the Ambeo
\footnote{Sennheiser Ambeo VR Mic. 
\todo{https://en-us.sennheiser.com/microphone-3d-audio-ambeo-vr-mic}}
characteristics ($R=0.015$ meter, $\alpha=0.5$). 
The generated audio has a duration of 60 seconds. 





\subsection{Recording}
Spherical isotropic noise has been rendered to a spherical loudspeaker layout with 25 \textit{Genelec 8040}. The loudspeakers are arranged into three azimuth-equidistant 8-speaker rings at inclinations $\theta = [\pi/4, \pi/2, 3\pi/4]$, plus one speaker at the zenith ($\theta=0$).
The different speaker distances to the center are delay- and gain-corrected, and the signal feeds are equalized to compensate for speaker coloration. The room has an approximate $T_{60}$ of 300 ms measured at the 1 kHz third-band octave. 
The spherical isotropic noise has been again created following the \textit{geometrical method}, encoding a number of uncorrelated noise plane waves in ambisonics with varying orders $L \in [1,5]$. Due to practical limitations related with the software, the minimum number of sources $N = 256$ for an accurate sound field reconstruction \cite{habets_comments_2010} could not be reached - instead, the analysis has been performed parametrically with $N = [8, 16, 32, 64]$.
For each value of $L$ and $N$, approximately 15 seconds of audio have been recorded with an Ambeo microphone located at the center of the speaker array.
Ambisonics decoding uses the AllRAD method \cite{zotter_all-round_2012}, passing through a spherical 64-point 10-design virtual speaker layout, and includes an imaginary speaker at the nadir ($\theta=\pi$). The decoding matrix uses \textit{in-phase} weights.



\subsection{Data processing and metrics}

The sampling rate of all signals is 48 kHz.
All frequency-domain results have been obtained by averaging their time-frequency representations over time.  
Ambisonics conversion is performed using \textit{Ambeo A-B converter} AU plugin, version 1.2.1.

Two error metrics are considered: the frequency-dependent squared error $\varepsilon(k)$, and the mean squared error $\bar{\varepsilon}$:

\begin{equation}
    \varepsilon(k) = |X_1(k) - X_2(k)|^2; \text{         }\bar{\varepsilon} = \frac{1}{K}{\sum_{k=1}^{K} |X_1(k) - X_2(k)|^2}
    \label{nmse}
\end{equation}



\section{Results and discussion}
\subsection{\label{subsec:results_aformat}A-Format}

\begin{figure}
	\includegraphics[width=\textwidth]{Figures/CoherenceEstimation/Figure1}
    \caption{\label{fig:Fig1}\textit{A-Format} coherence between microphone signals. Left: $\text{MSC}$ as a function of the frequency of theoretical, simulated and recorded \textit{((BLD,BRU)},  $L=5, N=64$) signals. Right: mean error $\bar{\varepsilon}$ of the recorded signals' $MSC$ \textit{(BLD,BRU)} compared to the simulated values, for all values of $L$ and $N$.}
\end{figure}


The coherence of the generated \textit{A-Format} signals is exemplified in Fig.~\ref{fig:Fig1} (left), which shows the $MSC$ between the capsule pair (\textit{BLD,BRU}) for the theoretical, simulated and recorded cases.
The theoretical coherence is derived from Eq.~(\ref{eq:closedform_msc}), while simulated and recorded MSC have been computed by Welch's method, using a \textit{hanning} window of 256 samples and 1/2 overlap.
The difference between theoretical and simulated coherence is negligible for practical applications.
However, there is a noticeable difference when compared to the recorded coherence. 
In general, the recorded $\text{MSC}$ follows the tendency of the simulated curve up to around 5 kHz.
Above this frequency, the recorded $MSC$ presents several spectral peaks, which might be partially explained by the interference of the microphone itself in the recorded sound field, and by the non-ideal directivity of the capsules.
The squared error $\varepsilon(k)$ with respect to the simulated curve is shown in Fig.~\ref{fig:Fig1} (left), while Fig.~\ref{fig:Fig1} (right) represents the same error averaged over frequency $\bar{\varepsilon}$ for different spatial resolution values of the diffuse field reproduction algorithm.
As expected, $\bar{\varepsilon}$ decreases with increasing values of $L$ and $N$.




\subsection{B-Format} 
\begin{figure}
	\includegraphics[width=\textwidth]{Figures/CoherenceEstimation/Figure2}
	\caption{\label{fig:Fig2} Estimated \textit{B-Format} coherence ($\Delta$) of a simulated diffuse sound field, as a function of the temporal averaging vicinity radius $r$. Left: $\Delta(k)$ for different values of $r$, with (coarse) and without (fine) application of radial filters. Right: mean and standard deviation of $\Delta(k)$ as a function of $r$.}
\end{figure}

In order to evaluate the dependency of $\Delta$ on the number of time frames used for averaging, the following procedure is presented.
The simulated \textit{A-Format} sound field has been transformed into the spherical harmonic domain, with and without the application of radial filters $\Gamma_m(kR)$. Then, $\Delta$ has been computed with Eq.~(\ref{eq:delta}) for exponentially growing values of $r$ between 1 (8 ms) and 2048 (10.92 s), where $r$ is the vicinity radius used for time averaging, and the number of time windows is given by $T = 2r+1$.
The time-frequency representation is derived by applying the STFT with the same window parameters as in Subsection \ref{subsec:results_aformat}.\\

Figure~\ref{fig:Fig2} (left) shows the great dependence of $\Delta$ on $r$.  The estimated coherence tends to the theoretical values with increasing values of $r$. This tendency is better appreciated in Fig.~\ref{fig:Fig2} (right): the curve asymptotically decreases to a value $\Delta_{min}\approx0$.
Another interesting observation comes from the frequency response of the curves. For all values of $r$, the coherence of the compensated \textit{B-Format} signal (with $\Gamma_m(kR)$) is roughly flat up to around 7 kHz, which approximately corresponds to the operational spatial frequency range of the microphone \cite{gerzon_design_1975}.
Above this value, the coherence response looses the flatness due to spatial aliasing. The response above the maximum frequency could be stabilized, if needed, by alternative diffuseness estimation methods \cite{politis_direction--arrival_2015}.
The coherence level differences along frequency are inversely proportional to $r$ --- the effect is better depicted by the standard deviation values (right).
The effect of the radial filters in the coherence measurement is also shown: for a given $r$, the shape of the coherence is always less flat if no filters are applied. Conversely, in this case, coherence values are always smaller for the same $r$. This effect might be explained taking into account the inter-channel coherence introduced by microphone and encoder imperfections in real scenarios \cite{schorkhuber_ambisonic_2017}.
As a remark, the comparison between Figs.~\ref{fig:Fig1} and \ref{fig:Fig2} provides evidence that the application of the spherical harmonic transform might be able to yield more accurate diffuseness estimations, due to a better signal conditioning \cite{epain_spherical_2016}.\\



\begin{figure}
	\includegraphics[width=\textwidth]{Figures/CoherenceEstimation/Figure3}
	\caption{\label{fig:Fig3}\textit{B-Format} coherence between microphone signals. Left: $\Delta$ of simulated and recorded ($L=5, N=64$) signals. Right: $\bar{\varepsilon}$ of the recorded signals coherence across all values of $L$ and $N$.}
\end{figure}

Figure~\ref{fig:Fig3} (left) shows the estimated coherence for the recorded sound field with $L=5$ and $N=64$, using a vicinity radius of $r=1024$ ($\approx$ 5 s).
The curve is centred around $\Delta=0.25$ and presents several spectral peaks, as in the \textit{A-Format} case. 
It is important to notice here that the deviations between the coherence of the simulated and the recorded sound fields are much stronger compared to those of Fig.~\ref{fig:Fig1}. 
This effect can be also appreciated in Fig.~\ref{fig:Fig3} (right): the mean squared error is around two orders of magnitude higher in \textit{B-Format}.
Nevertheless, similar as in Fig.~\ref{fig:Fig1} (right), $\bar{\varepsilon}$ decreases with increasing values of $L$ and $N$.
This behavior suggests that the deviations between the recorded and the simulated coherence can be to a large degree explained by the low spatial resolution of the reproduction system; given a higher number of loudspeakers, we expect that the reproduced diffuseness will tend to the theoretical expression.




\section{Conclusions}
The diffuseness of a sound field is an important parameter for several applications. In this work, two different metrics of diffuseness have been defined and measured with a tetrahedral microphone subjected to spherical isotropic noise.
The analysis shows, first, the impact of the time-averaging window length on the \textit{B-Format} diffuseness estimator.
This result might be useful for designing coherence estimators that are parametrized with respect to the length of the analysis window \cite{thiergart_diffuseness_2011}.
Second, the feasibility of diffuse sound field reproduction by a spherical loudspeaker array using ambisonics plane-wave encoding and the \textit{geometrical method} is studied. 
Results suggest that this approach is viable, given a sufficient spatial resolution; a quantification of the impact of the number of loudspeakers remains for future work.





\chapter{Sound Event Localization and Detection}
\chapter{Data generation and storage}
\label{chap:data}

\section{Introduction}
\label{sec:intro_data}

This chapter gathers several proposals related with the creation, storage and transmission of ambisonic data for research purposes. The main objective of the contributions described here is the support for the generation of parametrizable ambisonic datasets, using both synthetic and recorded materials, and specifically emphasizing the usage of Room Impulse Responses.\\ 

Most of the contributions listed here (and also most of the code developed for this thesis) have been implemented in Python. Indeed, Python has recently become one of the most used programming languages worldwide \cite{theoverflow, PYPL, TIOBE}; as shown also in Figure~\ref{fig:popularity}.\\

\begin{figure}
\label{fig_architecture}
  \centering
    \includegraphics[width=\textwidth]{Figures/DataGeneration/projections-1-1400x1200.png}
    \caption{2018 projections of future internet traffic for major programming languages. Adapted from \cite{theoverflow}.}
    \label{fig:popularity}
\end{figure}


One of the reasons behind this tendency shift is the popularity of the language among machine learning and data science communities, fields where Python holds the first place by usage \cite{githubblog}. 
Since data-driven paradigms currently conform the state-of-the-art of many applied sciences, including audio signal processing, the availability of convenience Python packages and libraries is therefore of the highest interest to the research community. 

It is important to remark the predominant position that Matlab has always had regarding scientific computing. Indeed, it is still the tool of choice for many researchers, and the availability of libraries is accordingly very high. 
But the aforementioned tendency shift towards Python causes, as a side effect, the lack of many tools developed in Matlab by the research community.

Although Matlab code can be called and executed from Python, in practice this approach is suboptimal under several criteria. A better solution in the long run is the effective port of the code towards native Python code. Some of the libraries presented in this Chapter are partially or totally  motivated by this scenario.

	

\section{MASP: a Python library for multichannel acoustic signal processing}

\subsection{Description}

The Multichannel Acoustic Signal Processing (\textit{MASP}) is a Python library consisting of a collection of methods related with acoustics and microphone array processing.
The library is mostly a transcoding from several Matlab libraries by A. Politis\cite{politis2016microphone, github_politis}. 
It can be conveniently installed using \textit{pip}.

\textit{MASP} implements a variety of methods for the simulation and analysis of reverberant acoustic scenes, with emphasis on microphone arrays with spherical geometries.
More specifically, MASP is structured in submodules, with the following structure :

\begin{description}

	\item [Array Response Simulator] Simulation of
spherical microphones:
	\begin{itemize}
		\item Rigid/open configurations.
		\item Scattering simulation.
		\item Arbitrary capsule distances, positions and directivities.
	\end{itemize}
	
	\item [Shoebox Room Model] Fast implementation of the Image Source Method \cite{imagemethod}:
	\begin{itemize}
		\item Convex 3D rooms.
		\item Arbitrary number of sources and receivers, with arbitrary positions, orientations and directivities.
		\item ISM expansion limited by order or time.
		\item Frequency-dependent wall absorption.
		\item RIR with spherical harmonic expansion.
	\end{itemize}
	
	\item [Spherical Array Processing] Transformation and analysis of signals measured with a spherical microphone array:
	\begin{itemize}
		\item A2B conversion with theoretical or measured filters.
		\item Signal-independent beamforming.
		\item Signal-dependent and adaptive beamforming.
		\item Direction of Arrival estimation.
		\item Diffuseness estimation.
	\end{itemize}
	
	\item [Spherical Harmonic Transform] Mathematical convenience tools.
\end{description}

The library implements a Unit Testing system, which numerically assesses the validity of the methods. 
More specifically, each function test calls the equivalent Matlab code under the hood. The numeric result is then sent back to Python, where
it is evaluated against the own result.\\

Two example applications of the library are shown in Figures~\ref{fig:array_response} and \ref{fig:sht_filters}.
In Figure~\ref{fig:array_response} , obtained with the Array Response Simulator package, the frequency response of a spherical microphone array to an impinging plane-wave with varying incidence angle is shown. The microphone array consists of a 2nd order supercardioid and a 3rd order hypercardioid, located at opposite directions of an open sphere, and both of them facing to the front direction. \\

\begin{figure}[h!]
\label{fig_architecture}
  \centering
    \includegraphics[width=\textwidth]{Figures/DataGeneration/array_response.png}
    \caption{Frequency response of an arbitrary spherical array.}
    \label{fig:array_response}
\end{figure}


One of the features of the Spherical Array Processing package is shown in Figure~\ref{fig:sht_filters}. The plot shows the evaluation of radial filters $\Gamma_n(kR)$ for an arbitrary spherical array, generated by inverting the theoretical response of the array \cite{Bertet2006}. 
The evaluation is performed following the metrics presented in the same paper, which compare spatial correlation, level difference and maximum amplification with respect to the ideal case. \\

\begin{figure}[h!]
\label{fig_architecture}
  \centering
    \includegraphics[width=1\textwidth]{Figures/DataGeneration/sht_filters.png}
    \caption{Evaluation of radial filters for an arbitrary spherical microphone array.}
    \label{fig:sht_filters}
\end{figure}

\subsection{Related software}

There exists another recent Python library which covers a similar scope: \textit{pyroomacoustics} \cite{scheibler2018pyroomacoustics}.
This framework provides an object-oriented interface with two main application scopes: allow RIR simulation of complex rooms based on the image source method, and provide a reference implementation of standard microphone array processing algorithms. 

Although some of the features are common to both libraries, there is a significant difference regarding their target usage. While \textit{MASP} primarily focuses on spherical geometries, \textit{pyroomacoustics} is more concerned about arbitrary room geometries and computational performance. 
Therefore, both libraries might be considered as complementary to some extent. 
A comparative list of their features is shown in Table~\ref{tab:masp_features}.



\begin{table}[th!]
\centering
\caption{Features of \textit{MASP} compared to \textit{pyroomacoustics}.}

\begin{tabular}{cccc}
  \toprule
Package & Feature &  MASP & PRA \\
\midrule
Shoebox & Convex 3D room         & \checkmark	& \checkmark \\
Room    & Non-convex 3D room     & -    			& \checkmark   \\
Model   & Arbitrary \#sources    & \checkmark   & \checkmark      \\
 & Arbitrary \#receivers, arrays & \checkmark   & \checkmark   \\
& ISM by max\_order                                 & \checkmark                              & \checkmark                              \\
& ISM by max\_time                                  & \checkmark                              & -                              \\
& Wall absorption                                   & \checkmark                              & \checkmark                              \\
& Frequency-dependent absorption                    & \checkmark                              & -                              \\
& Plot methods                                      & -                              & \checkmark                              \\
& RIR rendering                                     & \checkmark                              & \checkmark                              \\
& Audio simulation                                  & \checkmark                              & \checkmark                              \\
& Acoustic descriptor estimation                    & \checkmark                              & -                              \\
& Microphone orientation                            & \checkmark                              & -                              \\
& Custom microphone directivity                     & \checkmark                              & -                              \\
& RIR Spherical Harmonic Expansion                  & \checkmark                              & -                              \\

\midrule
Array   	& Rigid spherical arrays   & \checkmark                              & -     \\
Simulator 	& Arbitrary capsule geometries & \checkmark   & -  \\
           	& Recorded array IRs  & \checkmark                              & - \\
\midrule
Spherical & A2B conversion   & \checkmark      & -       \\
Processing  & Beamforming          		 & \checkmark      & \checkmark                              \\
 & Plane-wave decomposition         		 & \checkmark      & -                              \\
 & Nullformer                            & \checkmark      & -                              \\
& Adaptive Beamforming                   & \checkmark      & -                              \\
 & Adaptive Filtering             		 & -               & \checkmark                              \\
& DoA Estimation                         & \checkmark      & \checkmark                              \\
 & Diffuseness Estimation             	 & \checkmark      & -                              \\
 & Diffuse-field coherence               & \checkmark      & -                              \\
 & Blind Source Separation               & -               & \checkmark          \\                   
\bottomrule
 
\end{tabular}
\label{tab:masp_features}
\end{table}



\section{SOFA}

\subsection{Problem statement}

The availability of recorded room impulse responses is of great importance to many acoustic signal processing problems. 
Many different RIRs can be obtained from the same room, just by varying the position of the source and the receiver; when the number of source and receiver positions increases, the total amount of measurements increases geometrically. 
Besides that, the final format and organisation of the produced data (not only the RIR themselves, but also the source/position annotations) can be arbitrarily different when produced by different groups of people.

In order to overcome potential interoperatibility and reusability issues, the \textit{Spatially Oriented Format for Acoustics} (SOFA) convention \cite{majdak2013spatially}, also known as the \textit{AES-69} standard \cite{majdak2015aes69}, proposes a unified file format for the storage of IR-related data. 
Despite that SOFA was initially created with an emphasis on \textit{Head-Related Impulse Response} (HRIR) data, the framework that SOFA provides can be potentially applied to a variety of recording procedures and audio-related data. 
Such variety is associated with the concept of \textit{conventions}: a specific data structure designed to hold a concrete type of data or measurement. Some examples of widespread conventions might be \textit{SimpleFreeFieldHRIR} (for anechoic binaural measurements), \textit{SimpleHeadphoneIR} (intended for storing headphone impulse responses), or \textit{MultiSpeakerBRIR} (for binaural RIRs measured from loudspeaker arrays), to name a few of them.


\subsection{Ambisonics Directional Room Impulse Response as a SOFA convention}

Given the intrinsic spatial characterization capabilities of ambisonics, Gerzon proposed the technique as a potentially successful candidate format for acoustical heritage preservation, as early as 1975 \cite{gerzon1975recording}.

The increase in popularity of ambisonics since the beginning of the present century has turned this idea into reality; OpenAIRlib, a freely accesible dataset that gathers dozens of RIRs, might be a good example of it \cite{openair}.

In any case, the usage of recorded ambisonic RIRs is not limited to the field of acoustic heritage. Among others, the availability of such recordings has powered works in a variety of works, auralization \cite{postma2016virtual}, room acoustics analysis \cite{embrechts2015measurement,clapp2011investigations} and modelling \cite{romblom2017diffuse}, spatial audio synthesis \cite{coleman2017object} or source separation \cite{baque2016separation}. \\


In general, all publicly available ambisonic RIR measurements share some common approaches for describing and organizing the recorded data. 
For instance, recordings from different rooms are usually stored as separated folders. 
Each combination of emitter and receiver positions is often saved as an individual file, and the different spherical harmonics match the audio channels.
Moreover, it is also usual to provide a \textit{metadata} file, describing the different emitter and receiver positions, and potentially some information about the measurement setup, methodology, etc. Such files might be formatted as plain text or delimiter-separated files.
 
Despite the common approach, it can be easily foreseen that each database generated by a different individual or institution might potentially have a different naming convention, folder structure, file format, and so on. 
This is exactly the same situation that motivated the development of the SOFA conventions. 

On the other hand, the SOFA specification defines some criteria that must be fulfilled in order to propose a new convention \cite{sofaconventions}. These criteria are:
\begin{enumerate}
    \item Data must exist.
    \item Data can not be described by existing SOFA conventions.
    \item Relevant information about the data must be available.
\end{enumerate} 

Given that the described situation meets all requirements, the \textit{Ambisonics Directional Room Impulse Response} (AmbisonicsDRIR) convention has been therefore proposed as a new member of SOFA.\\

The technical specifications of the proposed convention in its current state (version 0.2) are available online \cite{ambisonicsdrir}. 
Despite a timid adoption of the convention, the AmbisonicsDRIR proposal has arisen interest in the community. At the moment of writing, the \textit{Standardisation Committee on AES-69 Standard} is discussing potential modifications to the SOFA file format required for the adoption of several new features; and the support for spherical harmonic-based measurements is among the topics for deliberation.


\subsection{pysofaconventions}

The situation described in Section~\label{sec:intro_data}, regarding the availability of acoustic signal processing libraries in the Python programming language, can be easily extended to the case of SOFA APIs.
 
The library \textit{pysofaconventions} has been created with the aim to provide an alternative to the existing Matlab/Octave and C/C++ implementations. 
For ease of installation, it is integrated in the standard python package manager, \textit{pip}. 

The current software version is 0.1.5. The library structure is inspired by the C++ implementation \cite{api_cpp}. 
It features all functionalities described by SOFA version 1.0, plus the proposed AmbisonicsDRIR convention. 
The implementation is based on extensive error checking, to ensure code consistency.
 
\textit{pysofaconventions} has gained a moderate amount of attraction. Among others, it is being used as a dependency for the \textit{Real-Time Spherical Microphone Renderer (ReTiSAR) for binaural reproduction}, developed by Chalmers University in collaboration with Facebook Reality Labs \cite{helmholz2019real}.


\section{Ambiscaper}


%
%Explain about mono files plus ambisonics IRs.
%
%
%\section{Recorded IRs}
%
%
%Impulse Responses (IRs) measurements constitute a compact way of representing the acoustic propertiesof a linear time-invariant system. When such measurements are performed in a specific room or enclosure, the so-called Room Impulse Responses (RIRs) are able to capture the intrinsic reverberation and acoustic characteristics of the enclosure, for which several methods have been developed these past years \cite{stan2002comparison}. Furthermore, it is possible to account for different emitter/receiver positions in the measurement, usually performing the measurement with a microphone array. In that case, this kind of measurements is referred to as Directional Room Impulse Responses (DRIRs) \cite{embrechts2005computation}. DRIRs have a wide range of applications: auralization \cite{embrechts2005computation}, room acoustics analysis \cite{embrechts2015measurement,clapp2011investigations} and modelling \cite{romblom2017diffuse}, spatial audio synthesis \cite{coleman2017object}, source separation and dereverberation \cite{baque2016separation}, acoustic heritage preservation \cite{gerzon1975recording,murphy2005multi}, etc. 
%
%
%\subsection{Ambisonics Recording}
%
%It is also possible to capture Ambisonics audio scenes by using specific recording devices. Spherical microphone arrays (also known as \textit{Ambisonics microphones}) are a type of microphone arrays in which the capsules are located around a spherical surface, presenting rotational symmetry. Such geometric arrangement allows the recorded signals to be transformed to the Ambisonics domain, by means of projection into the Spherical Harmonics basis, and further equalization with radial filters \cite{Bertet2006}.
%
%This process is known in the audio production domain as \textit{A-B Conversion}. All major spherical microphone array manufacturers provide tools for achieving this transformation \cite{soundfield,ambeo,zylia,tetra,em32}.
%
%
%\subsection{Ambisonics DRIRs}
%
%The intrinsic spatial capabilities of Ambisonics microphones might be applied to DRIR recording, as originally proposed by Gerzon in the context of acoustical heritage preservation \cite{gerzon1975recording}. In recent years, several datasets of Ambisonics DRIRs have been publicly released, such as the OpenAIR database \cite{murphy2010openair} or the set of  measurements performed in the scope of the S3A project \cite{coleman2015s3a,openair}. 
%
%\subsection{The SOFA Conventions}
%
%In general, the Ambisonics DRIR databases show a common approach for describing the measurements: given a specific room, usually represented as a folder, IR data consist of several multichannel audio files, with one audio channel per spherical harmonic, and one file per emitter/receiver combination. Furthermore, it is also usual to provide a \textit{metadata} file, describing the different emitter and receiver positions, and eventually some information about the measurement setup, methodology, etc. Such files might be formatted as plain text or delimiter-separated value files.
%
%Despite the common approach, it can be foreseen that each database generated by a different individual or institution might potentially have a different naming convention, folder structure, file format, and so on. This situation hinders data manipulation and exchange, and forces users to write ad-hoc parsers and algorithms for each specific database.
%
%The \textit{Spatially Oriented Format for Acoustics} (SOFA) is a file format designed for a consistent, standardized storage and manipulation of IR data \cite{majdak2015aes69}. The need for such standard arose from dealing with different databases of \textit{Head-Related Transfer Functions} (HRTFs), in a similar manner as the one mentioned for Ambisonics DRIRs.
%
%There are several SOFA conventions, each one addressing a particular type of IR measurement. In the case of Ambisonics DRIRs, given the data representation in existing databases, one could outline the following specificities:
% 
%\begin{itemize}
%    \item Presence of Ambisonics-related information (Ambisonics order, channel ordering and normalization)
%    \item Audio stored in the Ambisonics (spherical harmonics) domain  
%    \item Data structure support for different combinations of source and receiver positions
%\end{itemize}
%
%
%However, none of the existing SOFA conventions meets those requisites. The potentially first candidate by name, \textit{SingleRoomDRIR}, is limited to one source position per file. On the other hand, \textit{MultiSpeakerBRIR} allows for multiple sound sources, but restricts the number of receivers (microphone capsules) to two, as expected in a binaural recording. \textit{GeneralFIRE} is intended for "data which are too general to store in more specific conventions" \cite{generalfire}.
%
%
%\subsection{Convention Proposal}
%
%\subsubsection{Considerations}
%
%The SOFA specification defines some criteria that must be fulfilled in order to propose a new convention \cite{sofaconventions}. These criteria are:
%\begin{enumerate}
%    \item Data must exist.
%    \item Data can not be described by existing SOFA conventions.
%    \item Relevant information about the data must be available.
%\end{enumerate}
%
%
%As we already mentioned, existing databases of Ambisonics DRIRs can be found in OpenAIR and S3A databases (criterium 1). Criterium 2 has been discussed in the previous paragraph. All available Ambisonics DRIR datasets are accompanied by explanations, pictures, diagrams and related information (criterium 3). 
%
%\subsubsection{Specifications}
%
%We therefore propose a new SOFA convention, \textit{AmbisonicsDRIR}, designed for DRIRs measured with spherical microphone arrays, and presented in the Ambisonics domain. In other words, we propose to store the instantaneous $B_{mn}$ values for a given Ambisonics order $L$, provided that $S$ is a unit impulse (Eq. \ref{eq1}).
%
%The \textit{Listener} - as defined by the SOFA specification - is embodied by the spherical microphone array. The different \textit{Receivers} are the different Ambisonics Components $B_{mn}$, so that their number is fixed provided $L$ (more precisely, by following the relationship $R = (L+1)^2$), and their positions are not applicable. Furthermore, there might be many different \textit{Sources}, all of them omnidirectional and consisting of one only \textit{Emitter}. In that sense, \textit{M} represents the number of different \textit{Source} positions. 
%
%
%The proposed convention is based on \textit{GeneralFIRE}, with the following modifications:
%\begin{itemize}
%
%    \item Mandatory field \textit{GLOBAL:AmbisonicsOrder} (type \textit{double}, dimension \textit{I}, default \textit{1}). Indicates the order of the Spherical Harmonic expansion.
%    
%    \item Mandatory field \textit{DATA:ChannelOrdering} (type \textit{attribute}, default \textit{acn}). Describes the ordering of the different Ambisonics Channels. Must be one of: \textit{acn} or \textit{fuma}.
%    
%    \item Mandatory field \textit{DATAChannelNormalization} (type \textit{attribute}, default \textit{sn3d}). Describes the Ambisonics normalization convention used in the data. Must be one of: \textit{sn3d}, \textit{n3d}, \textit{fuma} or \textit{maxn}.
%    
%    \item Fields \textit{ReceiverPosition}, \textit{ReceiverPosition:Type} and \textit{ReceiverPosition:Units} are not needed, since it is assumed that \textit{Receivers} represent the Ambisonics channels. Furthermore, the values of \textit{R} (number of receivers) are defined by $R = (L+1)^2$. Accordingly, the values of \textit{GLOBAL:AmbisonicsOrder} and \textit{R} are valid only if they follow the given equation.
%    
%    \item Field \textit{DATA:Delay} is not mandatory.
%    
%    \item Relevant information about the microphone model and brand, Ambisonics encoding methodology, software used, etc, is recommended to be added into the field \textit{GLOBAL:Comment}.
%    
%\end{itemize}
%
%\subsection{Results}  
%
%As a preliminary result, we have extended the current SOFA C++  \cite{sofacpp} and Matlab \cite{sofamo} APIs to be fully compatible with the AmbisonicsDRIR convention. \todo{pysofaconventions}
%
%Furthermore, as a use-case, a selection of existing Ambisonics DRIRs have been transcoded to the proposed convention: \textit{Emmanuel Main Church} from S3A database \cite{coleman2015s3a}, and \textit{Heslington Church} and \textit{York Guildhall Council Chamber} from the OpenAIRlib \cite{openair}. Figure \ref{guildhall} shows a schematic diagram of the different \textit{Emitter/Receiver} positions at the \textit{York Guildhall Council Chamber} recordings. The data, as well as the tools used to perform the conversion, are available online under an open source license \cite{ambisonicsdrirexamples}.
%
%Finally, we must remark that the \textit{AmbisonicsDRIR} SOFA convention proposal is currently under discussion, and the described specifications are subject to change with upcoming versions. 
%
%% TODO: include some examples!!! about data dimensions etc
%
%\begin{figure}
%	\centering
%	\includegraphics[width=\textwidth]{Figures/DataGeneration/guildhall}
%	\caption{Source and Listener position diagram for the "York Guildhall Council Chamber" Ambisonics DRIR (original diagram attribution to \cite{guildhall} }
%	\label{guildhall}
%\end{figure}
%
%\subsection{Summary} 
%
%This document addresses the lack of compatibility among different databases of Ambisonics DRIRs. The present proposal consists in defining a new SOFA convention, specifically designed for Ambisonics DRIRs. That way we contribute to improve the ease of data manipulation and database interoperability. Software implementations and tools for automatic conversion are provided.
%
%
%
%\section{Simulated IRs}
%
%explain different methods and libraries.
%tell about masp.
%
%
%
%
%
%
%% ==================================================================
%\section{High-level scene description}
%
%\subsection{Motivation}
%
%\todo{introduce the topic}
%
%\todo{ARRANGE this introduction in a meaningful way}\\
%The work by Moore \cite{Moore2015} was the first to combine Intensity Vector statistics with DPD-test preprocessing. A similar approach is found on the \textit{Single Source Zones (SSZs)} algorithm, first presented by Pavlidi \cite{Pavlidi2015}. Other recent proposals, as for example the work by He \cite{He2017}, consider the local DOA variance as an estimator of reliability. 
%
%Source Localization based on IV statistics has been also successfully used as a preprocessing step for Source Separation tasks. The proposal of G\"unel \cite{Gunel2008}, which uses beamforming over the DOA estimation, was a pioneer work on this scope. Other similar approaches, which rely on DOA-based Time-Frequency (TF) masking, can be found on the works by Shujau \cite{Shujau2011}, Riaz \cite{Riaz2015} or Chen \cite{Chen2015}.
%
%
%
%\subsection{Evaluation data}
%\label{subsec:evaluationdata}
%
%Audio data used for evaluation in the works presented in Section \ref{subsec:algorithms} can be classified in  three main categories:
%
%
%\begin{itemize}
%
%    \item \textit{IR simulations:} Impulse Responses simulated by numerical methods, using software tools such as \textit{SMIR Generator} \cite{smir}. IR simulation provides the most flexible approach for evaluation, since it allows custom design and parametric analysis (algorithm evaluation as a function of reverberation time, source(s) distance(s), etc). However, the resulting audio data realism is restricted by the simulation algorithm, which is usually limited to empty \textit{shoebox} rooms.
%    
%    \item \textit{IR recordings:} Impulse Responses (IRs) of specific rooms recorded with an Ambisonics microphone, as first presented by Gerzon in the context of acoustic heritage preservation \cite{gerzon1975recording}. IR recordings do not provide the flexibility of IR simulations, but allow realistic captures of the room's acoustic properties, while maintaining the potential of using different source contents and positions. OpenAIRlib \cite{murphy2010openair} is the reference public repository for IRs, even including a dedicated Ambisonics IR section.
%
%    \item \textit{Audio recordings:} a sound scene recorded in a specific room, using an Ambisonics microphone and real sound sources. This approach  represents the  most  accurate  option  for algorithm evaluation in a real scenario. However, its cost and lack of flexibility makes this approach only attractive for last stages of algorithm  evaluation.
%
%\end{itemize}
%
%In the case of IR simulations and recordings, the Ambisonics IRs are convolved with anechoic monophonic recordings to produce the actual evaluation data. The anechoic recordings are usually taken from standard audio processing databases, such as TIMIT \cite{timit}. \\
%
%Table \ref{table:evaluationdata} summarizes the evaluation data types used in each work review in Section \ref{subsec:algorithms}, along with the audio content type and origin database (if any).    
%
%
%%% DOA ESTIMATION
%\begin{table}[htpb]
%    \centering
%    \footnotesize
%
%	\begin{tabular}{ p{2cm} p{2.8cm} p{2.5cm} p{2.3cm} }
%    \toprule
%    % \cmidrule(r){1-2}
%    
%    % \centering 
%    \textbf{Article} &
%    % \textbf{\textit{$L$}} &
%    \textbf{Evaluation Data Type} &
%    \textbf{Audio Content Type} &
%    \textbf{Database}\\
%    \midrule
%    
%    Thiergart \cite{thiergart_localization_2009}
%    % & 1 h
%    & Audio recording 
%    & Speech
%    & -\\
%    
%    Tervo \cite{Tervo2009}
%    % & 1 h
%    & Audio recording
%    & Noise, music
%    & -\\
%    
%    Jarret \cite{Jarrett2010} 
%    % & 4 
%    & IR simulation, Audio recording
%    & Noise
%    & -\\
%    
%    Nadiri \cite{Nadiri2014}   
%    % & 4 
%    & IR simulation, Audio recording
%    & Speech
%    & TIMIT \\
%    
%    Moore \cite{Moore2015}   
%    % & 4 
%    & IR simulation 
%    & Speech
%    & APLAWD \\
%    
%    Pavlidi \cite{Pavlidi2015}
%    % & 4
%    & IR simulation 
%    & Noise, speech
%    & -\\
%    
%    He \cite{He2017}   
%    % & 1 h
%    & IR simulation, Audio recording
%    & Speech
%    & TIMIT \\
%
%    
%    % Ding \cite{Ding2017}   
%    % % & 1 h
%    % & IR simulation \& Audio recording \\
%    
%    \midrule
%    
%    Gunel \cite{Gunel2008}
%    & IR recording
%    & Speech, music
%    & Music for Archimedes \\
%    
%    Shujau \cite{Shujau2011} 
%    & Audio recording
%    & Speech
%    & TIMIT \\
%    
%    Riaz \cite{Riaz2015}
%    & IR recording
%    & Speech, music
%    & Music for Archimedes \\
%    
%    Chen \cite{Chen2015}
%    & IR simulation, IR recording
%    & Speech
%    & TIMIT \\
%
%    \bottomrule
%    \end{tabular}
%    \caption{Summary of audio data used across Ambisonics-based Source Localization (above) and Source Separation (below)  algorithm proposals. \todo{update table? fix references}}
%    \label{table:evaluationdata}
%\end{table}
%
%
%
%Although algorithm evaluation in real scenarios provides the most realistic approach to the problem, this methodology has not been widely adopted due to its cost and lack of flexibility. As an example, none of the works using real recordings in Table \ref{table:evaluationdata} performs the evaluation with more than one  recording in each case. In contrast, when using IR-based  scenes, the number of audios evaluated are usually one or two magnitude orders greater. 
%
%However, it is important to notice the lack of public availability of evaluation data. None of the analyzed articles provide a way to access neither the used audio dataset, nor the groundtruth (position annotations in the case of Sound Localization, and original sound sources for Sound Separation). Only in the case of simulated IRs it is possible to partially replicate the experimental setup, since some of the parameters used in the simulation software are usually provided. Furthermore, the process of creation of the custom datasets (selecting an anechoic audio dataset, convolving with custom IRs, or performing real recordings) seems to be performed \textit{ad-hoc} on each article. 
%
%Taking into account the flexibility offered by IR-based scenes, it would be desirable to have a tool for automatic generation of reverberant Ambisonics scenes (and their associated groundtruth) for analysis purposes. Such tool would help the scientific community in several ways: reducing the amount of time dedicated to build custom datasets, reusing publicly available resources and recordings, and enhancing experiment reproducibility by making easier the exchange of datasets. Furthermore, the capacity of producing a big number of diverse audio scenes for analysis will help the algorithm design and early testing stages,and specially the training stage of machine learning-based algorithms. 
%
%
%\subsection{AMBISCAPER}
%\label{sec:ambiscaper}
%
%\todo{this has been downgraded to subsection. maybe place it at the introduction or something to find a more meaningful explanation}
%
%AmbiScaper \cite{ambiscaper} is a tool designed to provide a flexible way of creating complex Ambisonics sound scenes and their associated groundtruth, to be used in the context of Source Localization and Source Separation algorithms.
%AmbiScaper offers a high level control of the sound scene parameters, and provides an easy way of creating large datasets with custom characteristics.
%AmbiScaper is based on Scaper, a framework designed to generate ground truth information to train Sound Event Detection models \cite{Salamon2017}.
%
%\subsection{Sound scene description}
%\label{subsec:description}
%
%One of the main features of AmbiScaper is that all parameters of the sound scene can be specified in a non-deterministic way. In that sense, the parameters for each \textit{event} (sound source) are actually generated through a two-step process. First, in the \textit{Event Specification}, all parameters related to an event are defined in terms of statistical distributions. During the \textit{Event Instanciation}, the actual values for each parameter are then sampled from the statistical distributions.
%This two-step process allows the user to describe abstract \textit{templates} of sound scenes, rather than manually assigning values to parameters. Therefore, a single \textit{event specification} might produce potentially infinite different sound scenes. 
%
%
%
%\subsection{Architecture}
%\label{subsec:architecture}
%
%\begin{figure}
%\label{fig_architecture}
%  \centering
%    \includegraphics[width=\textwidth]{Figures/DataGeneration/figure_architecture_V2.png}
%    \caption{AmbiScaper architecture.}
%\end{figure}
%
%%%%%%%%%
%In order to generate a sound scene, AmbiScaper requires three different inputs: the original \textit{mono signals}, which will be the basis for the scene's audio content, an optional \textit{Ambisonics IR}, and the \textit{event specification}. 
%
%The process of dataset creation starts with the \textit{event instanciation}, as described in Section \ref{subsec:description}. Once all values are sampled, three different types of output are generated: the \textit{Ambisonics scene}, the \textit{instanciated mono signals} (the original mono signals after data augmentation and duration changes), and the \textit{annotations}, in the form of a JAMS file \cite{humphrey2014jams}, containing all information about the scene specification and the instanciated values. 
%
%AmbiScaper's architecture is depicted in Figure \ref{fig_architecture}.
%
%
%\subsection{Reverberation}
%
%When no reverberation is specified, AmbiScaper can generate anechoic sound scenes, which can be useful for baseline performance evaluation. In this case, there is no upper limit on the Ambisonics order of the rendered scene. Furthermore, the anoechic case allows for the specification of source \textit{spread} or apparent size through Ambisonics order downgrade \cite{carpentier2017ambisonic}.
%
%AmbiScaper supports the usage of recorded Ambisonics IRs, although it is currently limited to IRs from the S3A database \cite{coleman2015s3a}. The development of a standardized file format for Ambisonics IRs, which is being discussed at the moment of writing, will provide the flexibility to work with arbitrary Ambisonics IRs.   
%
%Lastly, AmbiScaper features the possibility of using simulated Ambisonics IRs, through a wrapper to \textit{SMIR Generator}. In this case, the reverberation model specifications might be defined as well in statistical terms, and the generated IRs are stored for evaluation purposes. A working copy of Matlab is required to run this option.
%
%When a reverberant sound scene is created, the specific Ambisonics IRs used on the scene are also provided as an output. Other research problems such as dereverberation or room reflection modelling might therefore benefit from such data.
%
%
%
%
%\subsection{AmbiScaper and experiment reproducibility}
%
%As mentioned in Section \ref{subsec:evaluationdata}, there is a generalized lack of publicly available datasets for Source Localization and Source Separation in the Ambisonics domain. Even when using general purpose audio/speech datasets (such as TIMIT), the actual reverberant Ambisonics evaluation data is usually not available.
%In that sense, the compatibility of AmbiScaper with public Ambisonics IR databases is a key aspect for reproducibility, since it allows for the reutilization of acoustical measurements in a systematic way.
%
%Furthermore, the output of the AmbiScaper dataset generation process is not limited to the actual dataset. In fact, as explained in Section \ref{subsec:architecture}, the resulting annotation file does not only contain the \textit{instanciation} (the actual values of each parameter in the sound scene), but also the \textit{specification} (the statistical distributions from which the instanciated values are sampled). In the scope of experiment reproducibility, the exchange of \textit{specification files} instead of actual audio files greatly reduces the storage capacitiy and bandwith required to transfer big databases. 
%
%AmbiScaper is implemented in the form of a Python package, publicly available through the \textit{Python Package Index} repository. 
%AmbiScaper is free software under the GPL license, easing the software adoption and the potential engagement of the scientific community with the development. 
%
%
%\subsection{Sample Dataset}
%\label{ssec:dataset}
%
%As an example of the potential capabilities of AmbiScaper,  we have created and published a dataset for the evaluation of source localization algorithms \footnote{https://zenodo.org/record/1186907 \todo{link}}.
%
%The dataset contains 300 first order Ambisonics sound scenes, each one containing from one to three static sound sources (be they simultaneous or not), with different gains and SNRs, and placed at random positions around the sphere. Each sound scene has a duration between 1 and 2 seconds.
%
%The sources are randomly chosen from a subset of the Anechoic OpenAirlib database, which mostly contains recordings from baroque musical instruments.
%The AudioBooth IR database from S3A has been used for all scenes.
%It features SoundField recordings from an acoustically treated room with a geodesic dome structure, to which 17 speakers have been attached.
%
%
%
%\subsection{CONCLUSIONS AND FUTURE WORK}
%\label{sec:conclusions}
%
%AmbiScaper is a tool designed for easy dataset creation and exchange, in the context of reverberant Ambisonics Source Localization and Source Separation. It responds to the lack of public datasets for algorithm design, evaluation and reproducibility. An example dataset generated with AmbiScaper, and its analysis with state-of-the-art Source Localization algorithms are also provided.
%Further studies in Ambisonics IR position interpolation would allow creating non-static audio scenes, which might be an interesting feature. 
%Another feature in progress is the support for a standardized Ambisonics IR format \cite{perez2018ambisonics}, which will ease data reusability and provide the potential to cover a big variety of acoustic scenarios. 
%
%
%
%
%
%
%

\chapter{Conclusions}


\section{Summary of Contributions}

In this thesis we have presented our contributions to different components of an ambisonics analysis and generation framework, with a focus on reproducibility and portability to real-world scenarios.
The main scientific objectives of this thesis, as they were described in Section \ref{sec:objectives}, are: 
 

\begin{enumerate}
	\item The development of methods to support and improve the characterization of acoustic parameters.
	\item The research on parametric-based methodologies for sound event localization and detection.
	\item The contribution in the generation and storage of annotated ambisonic datasets.
\end{enumerate}

In what follows, we summarize the main contributions of the present thesis, both in the academic and software scopes.

%  \begin{itemize}

%  	 \item Blind reverberation time estimation
\newpage
\paragraph{Blind reverberation time estimation}
Chapter~\ref{chap:rt60} presents a novel methodology for the blind estimation of reverberation time from ambisonic audio. The method is based on two main steps: first, dereverberation using a multichannel auto-recursive model (MAR), and second, estimation of the  filter from reverberant and dereverberated signals. The actual reverberation time value is estimated from the energy decay of the estimated filter. 

The proposed system is the first attempt in the literature to address the blind reverberation time estimation problem specifically for ambisonic signals. Compared with a state-of-the-art monophonic estimator, our method is able to improve in all the evaluation metrics under consideration.


\paragraph{Coherence estimation}
In Chapter~\ref{chap:coherence}, we have characterised the response of tetrahedral microphones to isotropic noise field, which is one of the most used models for diffuse sound. Furthermore, the capabilities of a spherical loudspeaker array with respect to the reconstruction of diffuse sound fields using ambisonics are also quantitatively analyzed.

\paragraph{Sound event localization and detection}
Chapter~\ref{chap:seld2019} describes an algorithm for sound event localization and detection (SELD), developed in the context of the DCASE 2020 challenge. The method estimates the localization and temporal activity of the sound events based on a particle filter that tracks event trajectories obtained from the parametric analysis of the ambisonic sound field. Each event is assigned to a sound class by a machine learning classifier that uses low- and mid- level audio features. 

Results show a significant performance increase in all evaluation metrics under consideration, compared with a state-of-the-art deep learning baseline.
This suggests that our approach, substantially different to the baseline and the majority of state-of-the-art methods, represents a feasible alternative in situations with low-complexity or small database constraints. 


\paragraph{Data generation and management}
Finally, the thesis contributions to more practical aspects are presented in Chapter~\ref{chap:data}. Those contributions comprise two software libraries written in Python: one of them focused on spherical microphone array and acoustic simulation, and another one implementing the SOFA standard, which has also been revised and modified for allowing the representation of ambisonic data. 
Finally, a novel software tool for the procedural creation of annotated reverberant ambisonic datasets has ben also presented.

%  	\begin{description}
%  		\item [1. Parameter estimation] Novel technique for blind RT60 estimation of ambisonic recordings from autoregressive models.
%	\end{description}
%
%  
%  	\item Coherence estimation (Chapter~\ref{chap:coherence}):
%  	\begin{description}
%  		\item [1. Parameter estimation] Contribution to the characterization of coherence with tetrahedral microphones (the most common spherical arrangement).
%	\end{description}
%
%
%  	\item Sound Event Localization and Detection (Chapter~\ref{chap:seld2019}):
%  	\begin{description}
%  		\item [2. Scene Description]: Novel state-of-the-Art methodology for Sound Event Localization and Detection	.
%  	\end{description}
%  	
%  	\item Data generation and storage (Chapter~\ref{chap:data}):
%  		\begin{description}
%  			\item [3. Data Generation]: Library for acoustic simulation and spherical microphone array processing.
%  			\item [3. Data Generation]: Proposal and implementation of a file convention for the storage of recorded ambisonic RIRs.
%  			\item [3. Data Generation]: Novel tool for high-level description and generation of of ambisonic datasets.
%
% 		\end{description}
%
%  \end{itemize}
  
   
%Figure~\ref{fig:scheme1_numbers2}, which is a copy of Figure~\ref{fig:scheme1_numbers}, 
% has been included here again in order to help the contextualization of the contributions. 
%
%  \begin{figure}[h!]
%  \includegraphics[width=\textwidth]{Figures/Introduction/SCHEME1_NUMBERS.png}
%  \caption{General scheme of the B-Format audio generation and analysis framework, including the thesis contributions in form of Chapter numbers.}
%  \label{fig:scheme1_numbers2}
%\end{figure}




\section{Future Work}

This thesis has tackled several research problems associated with the analysis of ambisonic recordings, making use primarily of the parametric sound field modelling analysis.
The presented techniques improve existing state-of-the-art methodologies, or present novel approaches for known research problems, which in turn bring new research questions. \\

%Chapter~\ref{chap:rt60} presents a novel methodology for the blind estimation of reverberation time from ambisonic audio. The method is based on two main steps: first, dereverberation using a multichannel auto-recursive model (MAR), and second, estimation of the  filter from reverberant and dereverberated signals. The actual reverberation time value is estimated from the energy decay of the estimated filter. 
%
%The proposed system is the first attempt in the literature to address the blind reverberation time estimation problem specifically for ambisonic signals. Compared with a state-of-the-art monophonic estimator, our method is able to improve in all the standard evaluation metrics under consideration.

A novel blind reverberation time estimation method for first-order ambisonic recordings is introduced in Chapter~\ref{chap:rt60}.
Among others, the method could be straightforwardly extended to higher order ambisonic signals, which might still improve the reported results due to the availability of many more audio channels. 
Moreover, the usage of online MAR methods would enable the possibility of analyzing sound scenes with moving sources; the statistical time-invariance property of late reverberation supports this hypothesis.
Finally, it is important to mention that the proposed method is resource-intensive. An analysis of the trade-off between computation time and evaluation performance, mostly dependent on the estimation filter length, remains to be done.  
\newpage
Given the current interest in the field of augmented reality, new related research topics emerge. One of them, which has been recently baptized as \textit{acoustic matching} \cite{su2020acoustic}, deals with the analysis of acoustic properties of real enclosures, with the aim of later introduction of virtual elements whose reverberation would match real conditions. 
The application of our method to the acoustic matching problem is straightforward: given an ambisonic recording with a target reverberation, estimate its reverberation time and synthesize a reverberant tail with the target energy decay; early reflections might be generated by various methods, including physical models or perceptually motivated approaches. 
We can foresee a growing interest on the topic in the near future; our contribution might help to establish the foundation of a new family of methods. \\


%In Chapter~\ref{chap:coherence}, we have characterised the response of tetrahedral microphones to isotropic noise field, which is one of the most used models for diffuse sound. Furthermore, the capabilities of a spherical loudspeaker array with respect to the reconstruction of diffuse sound fields using ambisonics are also tested. 
Regarding the diffuse field characterization performed in Chapter~\ref{chap:coherence}, an immediate extension of the work would include the study of different spherical microphone array geometries, from the ones that are commercially available.
The usage of different models of diffuse fields, specially extending to the anisotropic case \cite{alary2019assessing}, might also constitute an interesting research continuation direction. 
Both cases could be also applied to the experiment of diffuse sound field reconstruction using loudspeaker arrays. \\

%in Future work in this direction might include the characterisation of different spherical microphone array geometries, and different types of diffuse field models. Additionally, the experiment on loudspeaker array diffuse field reconstruction could be easily extended, with the aim of including more . \\


%Chapter~\ref{chap:seld2019} describes an algorithm for sound event localization and detection (SELD), developed in the context of the DCASE 2020 challenge. The method estimates the localization and temporal activity of the sound events based on a particle filter that tracks event trajectories obtained from the parametric analysis of the ambisonic sound field. Each event is assigned to a sound class by a machine learning classifier that uses low- and mid- level audio features. Results on the cross-validation development set show a significant performance increase in most evaluation metrics, compared with a state-of-the-art deep learning baseline. This result shows that our approach, which is substantially different to the baseline and the majority of state-of-the-art methods, represents an alternative in specific situations with low-complexity or small database constraints. 

The wide scope of the SELD problem, as described in Chapter~\ref{chap:seld2019}, allows for a wide range of possibilities regarding a potential follow-up of the proposed method.
For instance, one of the major problems of our algorithm is the inaccurate parametric estimation when two events are simultaneously active. Although it is a known problem in the literature \cite{epain2016spherical} , a successful solution in the given context remains still to be explored.
\newpage
Another source of potential improvements is the refinement of the particle filter applied to this specific task. A deeper understanding of control theory, as well as collaborations with experts on the field, might bring a noticeable improvement on the overall system performance.

The performance of the event classifier might be also further analyzed. Although in this case we opted for a classical feature-based machine learning approach, different methods and architectures could be compared, including more modern deep-learning approaches.\\

%Finally, the thesis contributions to more practical aspects are presented in Chapter~\ref{chap:data}. Those contributions comprise two software libraries written in Python: one of them focused on spherical microphone array and acoustic simulation, and another one implementing the SOFA standard, which has also been revised and modified for allowing the representation of ambisonic data. Finally, a software for the procedural creation of annotated reverberant ambisonic datasets has ben also presented. 

Finally, we discuss briefly on the practical contributions described in  Chapter~\ref{chap:data}.
Apart from the straightforward task of software maintenance, the engagement of the research community with the usage and eventual contribution of the libraries would represent a desired situation in the near future.
Moreover, the proposed file format convention is being currently discussed by the \textit{Standardisation Committee on AES-69 Standard}, which can be considered a great achievement of the original proposal. 
The results of the discussion might lead to the addition of a modified version of our proposal into version 2 of the \textit{AES-69 Standard}. 


In the medium term, the application at commercial level of some of the methods  described in this thesis would be a highly desirable outcome of our research. Such application would probably imply a re-implementation at the software level, intended for compatibility with the workflows and formats used in the VR/AR content production industry. 


\newpage
\section{List of Contributions}

In the following list we show the main scientific contributions, as main author, related to this dissertation:

\begin{itemize}

	\item Peer-reviewed journal articles
	
	\textbf{"Analysis of spherical isotropic noise fields with an A-Format tetrahedral microphone"}.
	\underline{A. P\'erez-L\'opez} and N. Stefanakis.
	\textit{The Journal of the Acoustical Society of America 146.4 (2019): EL329-EL334.}

	\item Peer-reviewed conference articles

	\textbf{"Blind reverberation time estimation from ambisonic recordings"}.
	\underline{A. P\'erez-L\'opez}, A. Politis and E. G\'omez.
	Submitted to \textit{IEEE 22nd International Workshop on Multimedia Signal Processing, 2020.}\\

	\textbf{"PAPAFIL: a low complexity sound event localization and detection method with parametric particle filtering and gradient boosting".}
	\underline{A. P\'erez-L\'opez} and R. Iba�ez-Usach.
	Submitted to \textit{Detection and Classification of Acoustic Scenes and Events 2020 Workshop (DCASE2020)}.\\
	
	\textbf{"A hybrid parametric-deep learning approach for sound event localization and detection".}
	\underline{A. P\'erez-L\'opez}, E. Fonseca and X. Serra.
	In \textit{Proceedings of the Detection and Classification of Acoustic Scenes and Events 2019 Workshop (DCASE2019)}.
	
	\textbf{"Ambiscaper: A tool for automatic generation and annotation of reverberant ambisonics sound scenes".}\\
	\underline{A. P\'erez-L\'opez}.
	In \textit{Proceedings of the 16th International Workshop on Acoustic Signal Enhancement (IWAENC). IEEE, 2018.}
	
	
	\item Conference engineering briefs


	\textbf{"Ambisonics directional room impulse response as a new convention of the spatially oriented format for acoustics".}
	\underline{A. P\'erez-L\'opez} and J. De Muynke.
	In \textit{Proceedings of the 144th Audio Engineering Society Convention. Audio Engineering Society, 2018.}\\
	
	\textbf{"pysofaconventions, a Python API for SOFA".}
	\underline{A. P\'erez-L\'opez}.
	In \textit{Proceedings of the 148th Audio Engineering Society Convention. Audio Engineering Society, 2020.}\\
	
	\textbf{"A Python library for multichannel acoustic signal processing".}
	\underline{A. P\'erez-L\'opez} and A. Politis.
	In \textit{Proceedings of the 148th Audio Engineering Society Convention. Audio Engineering Society, 2020.}\\
	
\end{itemize}


Moreover, although not strictly aligned with the research direction of this thesis, the author has supervised the following publication:\\

	 \textbf{"Sound event localization and detection based on CRNN using dense rectangular filters and channel rotation data augmentation".}
	F. Ronchini, D. Arteaga and \underline{A. P\'erez-L\'opez}.
	Submitted to \textit{Detection and Classification of Acoustic Scenes and Events 2020 Workshop (DCASE2020)}.\\


\newpage
\section{List of Software Resources}


As a result of the development of this thesis, the following open-source libraries and repositories have been produced. All of them are freely available through the author's GitHub page \cite{andresgithub} under open-source licenses. 


\begin{itemize}

	\item Software tools:

	
	\href{https://github.com/andresperezlopez/masp}{masp: Multichannel Acoustic Signal Processing library} 
		Tools for acoustic simulation and spherical array processing. \\
		 
	\href{https://github.com/andresperezlopez/pysofaconventions}{pysofaconventions} Implementation of the SOFA convention in Python.\\ 
	
	\href{https://github.com/andresperezlopez/AmbisonicsDRIR}{AmbisonicsDRIR}	 Ambisonic SOFA Convention proposal.\\

	\href{https://github.com/andresperezlopez/ambiscaper}{AmbiScaper} Tool for automatic generation of annotated ambisonic datasets.\\

	\item Method implementations: 

	\href{https://github.com/andresperezlopez/ambisonic_rt_estimation}{ambisonic\_rt\_estimation} Contains the code implementing the methods described in Chapter~\ref{chap:rt60}.\\
	
	\href{https://github.com/andresperezlopez/DCASE2020}{DCASE2020} Full code implementing the SELD algorithm from Chapter~\ref{chap:seld2019}.\\
	
	\href{https://github.com/andresperezlopez/DCASE2019_task3}{DCASE2019\_task3} 
		Implementation of the method submitted to DCASE2019 (not included in this thesis). \\


\end{itemize}






%ADD BIBLIOGRAPHY
\bibliography{Bibliography/CoherenceEstimation, Bibliography/SELD.bib, 
	Bibliography/Data.bib, 
	Bibliography/AmbisonicsDRIR.bib, 
	Bibliography/ReverberationTimeEstimation.bib,
	Bibliography/Introduction.bib, 
	Bibliography/ScientificBackground.bib,
	Bibliography/Conclusion.bib}


\backmatter
\printindex

\end{document}


%NUMBER OF THE EXTERNAL PAGE EXCEPT IN THE FIRST PAGE OF EACH CAPITAL
%\usepackage{fancyhdr}
%\pagestyle{fancy}
%\fancyfoot{}
%\fancyfoot[RO]{\thepage}
%\fancyfoot[LE]{\thepage}
%
%
%%MULTIPLE INDEX
%%In the preamble
%\usepackage{multind}
%\makeindex{authors}
%%Introduction to form entries
%\index{authors}{Einstein}
%%Situation of the Index
%\printindex{authors}{Author index}
%%The \ usepakage {makeidx} \ makeindex \ printindex commands must be removed
%%You need to exacute from the command line makeindex authors